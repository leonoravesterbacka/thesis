\chapter{Results}\label{sec:results}
Once all the background prediction techniques are in place, it is time to compare the predictions to the observed yield in the signal region kinematic variables. 
The \ptmiss is the detector observable whcih provide the strongest discriminator the various SUSY signals considered and the SM backgrounds. 
To visualize the results, the \ptmiss of the stacked predicted SM backgrounds are overlayed with the observed data and the final uncertainties, including statistical and systematic components, are represented as shaded band. 
In addition to the main observable, the \ptmiss, multiple other observables are scrutinized as a means to ascertain that significant discrepancies between the data and the simulation are not present in the signal region.
As will be seen, there is a good agreement between the predicted backgrounds and the observed data, indicating the absence of a significant excess. 
In absence of an excess in data, the subsequent chapter will describe the procedure used to interpret the results in terms of models of new physics. 
\section{Results on strong SUSY production}
\subsection{Strong On-Z search}
The results of the strong on-Z search for GMSB in the gluino induced process is presented in Table \ref{tab:resultsOnZAB} and Table \ref{tab:resultsOnZC}. 
The largest background is stemming from Drell-Yan process in all signal regions and \ptmiss regions. 
There is a good agreement between the prediction and the observed data, well within the systematic uncertatinties.  
\begin{table}[ht!]
\def\arraystretch{1.2}
\setlength{\belowcaptionskip}{6pt}
\small                             
\centering
\caption{Standard model background predictions compared to observed yield in the strong on-Z signal regions (SRA-SRB). }
\label{tab:resultsOnZAB}
\begin{tabular}{l c c c }
\hline \hline
 \multicolumn{4}{c}{SRA (b-veto)} \\
 \ptmiss [GeV] & 100--150              & 150--250                       & $>250$ \\ \hline
 DY+jets        & 13.6$\pm$3.1         & 2.5$\pm$0.9                    & 3.3$\pm$2.4 \\
 FS bkg.           & $0.4^{+0.3}_{-0.2}$  & $0.2^{+0.2}_{-0.1}$            & $0.2^{+0.2}_{-0.1}$  \\
 $\PZ+\nu$          & 0.8$\pm$0.3          & 1.4$\pm$0.4                    & 2.4$\pm$0.8 \\
 Total background           & 14.8$\pm$3.2 & 4.0$\pm$1.0            & 5.9$\pm$2.5 \\
 Data          & 23                   & 5                              & 4 \\ \hline
\hline \multicolumn{4}{c}{SRA (b-tag)} \\
\ptmiss [GeV] & 100--150              & 150--250                       & $>250$ \\ \hline
DY+jets        & 8.2$\pm$2.1          & 1.2$\pm$0.5                    & 0.5$\pm$0.3 \\
FS bkg.           & 2.3$\pm$0.8  & $1.7^{+0.7}_{-0.6}$            & $0.1^{+0.2}_{-0.1}$  \\
$\PZ+\nu$          & 1.9$\pm$0.4          & 2.0$\pm$0.5                    & 1.8$\pm$0.6 \\
Total background           & 12.4$\pm$2.3 & 4.9$\pm$1.0            & 2.5$\pm$0.7 \\
Data          & 14                   & 7                              & 1 \\ \hline
\hline \multicolumn{4}{c}{SRB (b-veto)} \\
\ptmiss [GeV] & 100--150              & 150--250                       & $>250$ \\ \hline
DY+jets        & 12.8$\pm$2.3         & 0.9$\pm$0.3                    & 0.4$\pm$0.2 \\
FS bkg           & $0.4^{+0.3}_{-0.2}$  & $0.4^{+0.3}_{-0.2}$            & $0.1^{+0.2}_{-0.1}$  \\
$\PZ+\nu$          & 0.3$\pm$0.1          & 0.7$\pm$0.2                    & 1.2$\pm$0.4 \\
Total background           & 13.6$\pm$2.4 & 2.0$\pm$0.5            & 1.6$\pm$0.4 \\
Data          & 10                   & 4                              & 0 \\ \hline
\hline \multicolumn{4}{c}{SRB (b-tag)} \\
\ptmiss [GeV] & 100--150              & 150--250                       & $>250$ \\ \hline
DY+jets        & 7.7$\pm$3.2          & 4.0$\pm$3.4                    & 0.1$\pm$0.1 \\
FS bkg.           & $1.4^{+0.6}_{-0.5}$  & $1.1^{+0.5}_{-0.4}$            & $0.2^{+0.2}_{-0.1}$  \\
$\PZ+\nu$          & 2.0$\pm$0.5          & 2.3$\pm$0.6                    & 1.0$\pm$0.3 \\
Total background           & 11.1$\pm$3.3 & $7.4^{+3.5}_{-3.4}$            & $1.3^{+0.4}_{-0.3}$ \\
Data          & 10                   & 5                              & 0 \\ \hline\hline
\end{tabular}
\end{table}   
 
\begin{table}[ht!]
\def\arraystretch{1.2}
\setlength{\belowcaptionskip}{6pt}
\small                             
\centering
\caption{Standard model background predictions compared to observed yield in the strong on-Z signal regions (SRC). }
\label{tab:resultsOnZC}
\begin{tabular}{l c c }
\hline \hline
 \multicolumn{3}{c}{SRC (b-veto)} \\
\ptmiss [GeV] & 100--150              & $>150$ \\ \hline
DY+jets        & 1.2$\pm$0.4          &  0.1$\pm$0.1  \\
FS bkg.           & $0.4^{+0.3}_{-0.2}$  &  $0.1^{+0.2}_{-0.1}$   \\
$\PZ+\nu$          & 0.1$\pm$0.1          & 0.5$\pm$0.2  \\
Total background           & 1.7$\pm$0.5  &  $0.7^{+0.3}_{-0.2}$  \\
Data          & 4                    &  0  \\ \hline
\hline \multicolumn{3}{c}{SRC (b-tag)} \\
\ptmiss [GeV] & 100--150              &  $>150$ \\ \hline
DY+jets        & 0.1$\pm$0.4          & 0.0$\pm$0.3 \\
FS bkg.           & $0.0^{+0.1}_{-0.0}$  &0.3$\pm$0.2  \\
$\PZ+\nu$          & 0.6$\pm$0.2          & 0.6$\pm$0.2 \\
Total background           & 0.8$\pm$0.5  & $0.9^{+0.5}_{-0.4}$  \\
Data          & 2                    &  2  \\ \hline\hline
\end{tabular}
\end{table} 

\subsection{Edge search}
The edge search is binned in the invariant mass of the SF leptons, instead of in \ptmiss. 
Seven \mll bins are further split into two categories according to the \ttbar likelihood discriminant, resulting in a total of 14 signal region bins. 
The predicted SM backgrounds are compared to the observed data in Table \ref{tab:edgeResults} and in Figure \ref{fig:resultsEdge} as a graphical representation. 

\begin{table}[!hbtp]
\renewcommand{\arraystretch}{1.2}
\setlength{\belowcaptionskip}{6pt}
\small
\centering                             
\caption{Predicted and observed yields in each bin of the edge search counting experiment. The uncertainties shown include both statistical and systematic sources.}
\label{tab:edgeResults}
\begin{tabular}{ c  c  c  c  c  c}
\hline
\hline
\mll range [GeV] & FS bkg.& DY+jets & $\PZ+\nu$  & Total background & Data\\
\hline
\multicolumn{6}{c}{\ttbar-like}  \\
\hline
20--60    &  291$^{+21}_{-20}$    & 0.4$\pm$0.3   & 1.4$\pm$0.5  &  293$^{+21}_{-20}$ & 273 \\
60--86    &  181$^{+16}_{-15}$    & 0.9$\pm$0.7   & 8.8$\pm$3.4  &  190$^{+16}_{-15}$ & 190 \\
96--150   &  176$^{+15}_{-14}$    & 1.1$\pm$0.9   & 6.0$\pm$2.4  &  182$^{+16}_{-15}$ & 192 \\
150--200  &  73$^{+10}_{-9}$      & 0.1$\pm$0.1   & 0.4$\pm$0.2  &  74$^{+10}_{-9}$ & 66 \\
200--300  &  46.9$^{+8.4}_{-7.3}$ & $< 0.1$       & 0.3$\pm$0.1  &  47.3$^{+8.4}_{-7.3}$ & 42 \\
300--400  &  18.5$^{+5.7}_{-4.5}$ & $< 0.1$       & $< 0.1$      &  18.6$^{+5.7}_{-4.5}$ & 11 \\
$> 400$   &  4.3$^{+3.4}_{-2.1}$  & $< 0.1$       & $< 0.1$      &  4.5$^{+3.4}_{-2.1}$ & 4 \\
\hline
\multicolumn{6}{c}{Not-\ttbar-like}   \\
\hline
20--60    &  3.3$^{+3.2}_{-1.8}$    & 0.7$\pm$0.5   & 1.4$\pm$0.5  &  5.3$^{+3.3}_{-1.9}$ & 6 \\
60--86    &  3.3$^{+3.2}_{-1.8}$    & 1.6$\pm$1.3   & 6.9$\pm$2.7  &  11.8$^{+4.4}_{-3.5}$ & 19 \\
96--150   &  6.6$^{+3.9}_{-2.6}$    & 1.9$\pm$1.5   & 6.8$\pm$2.7  &  15.3$^{+5.0}_{-4.1}$ & 28 \\
150--200  &  5.5$^{+3.7}_{-2.4}$    & 0.2$\pm$0.3   & 0.7$\pm$0.3  &  6.4$^{+3.7}_{-2.4}$ & 7 \\
200--300  &  3.3$^{+3.2}_{-1.8}$    & 0.2$\pm$0.2   & 0.5$\pm$0.2  &  3.9$^{+3.2}_{-1.8}$ & 4 \\
300--400  &  3.3$^{+3.2}_{-1.8}$    & $< 0.1$       & 0.2$\pm$0.1  &  3.5$^{+3.2}_{-1.8}$ & 0 \\
$> 400$   &  1.1$^{+2.5}_{-0.9}$    & $< 0.1$       & 0.4$\pm$0.2  &  1.6$^{+2.5}_{-0.9}$ & 5 \\
\hline
\hline
\end{tabular}
\end{table}

\begin{figure}[htbp!]
\begin{center}
\includegraphics[width=0.9\textwidth]{images/results/Figure_005.pdf}
\caption{Results of the counting experiment of the edge search. For each SR, the number of observed events, shown as black data points, is compared to the total background estimate.  The hashed band shows the total uncertainty in the background prediction, including statistical and systematic sources.}
\label{fig:edgeResults}
\end{center}
\end{figure}                                                                                                                                                               

The main SM background in the edge search is due to the flavor symmetric processes. 
As the uncertainty on the flavor symmetric background prediction technique is driven by the statistical uncertatinty in the number of events in the OF control sample, this becomes the dominant uncertainty in the high-mass and non-\ttbar-like regions where the event counts are low. 
There is good agreement between prediction and observation for all SRs. 
There is a deviation observed in the not-\ttbar-like region for masses between 96 and 150\GeV, with an excess corresponding to a local significance of 2.0 standard deviations.

The dilepton mass distributions and the results of the kinematic fit are shown in Fig.~\ref{fig:Fit_data_H1}.
Table~\ref{tab:fitResults} presents a summary of the fit results.
A signal yield of $61 \pm 28$ events is obtained when evaluating the signal hypothesis in the baseline SR,
with a fitted edge position of $144.2^{+3.3}_{-2.2}\GeV$. This is in agreement with the upwards fluctuations in the mass
region between 96 and 150\GeV in the counting experiment and corresponds to a local significance of 2.3 standard deviations.
To estimate the global $p$-value~\cite{Gross:2010qma} of the result, the test statistic $-2\ln Q$,
where $Q$ denotes the ratio of the fitted likelihood value for the signal-plus-background
hypothesis to the background-only hypothesis, is evaluated on data and compared to the respective quantity on a large sample
of background-only pseudo-experiments where the edge position can have any value. The resulting $p$-value is interpreted as the one-sided tail probability
of a Gaussian distribution and corresponds to an excess in the observed number of events compared to the SM background prediction
with a global significance of 1.5 standard deviations.

\begin{figure}[!hbtp]
\centering
\includegraphics[width=0.42\textwidth]{images/results/Figure_006-a.pdf}
\includegraphics[width=0.42\textwidth]{images/results/Figure_006-b.pdf}
\caption{Fit of the dilepton invariant mass distributions to the signal-plus-background hypothesis in the ``Edge fit'' SR from Table~\ref{tab:edgeSR}, projected on the same flavor (left) and opposite flavor (right) event samples. The fit shape is shown as a solid blue line. The individual fit components are indicated by dashed and dotted lines. The FS background is shown with a black dashed line. The Drell-Yan background is displayed with a red dotted line. The extracted signal component is displayed with a purple dash-dotted line. The lower panel in each plot shows the difference between the observation and the fit, divided by the square root of the number of fitted events.}
\label{fig:Fit_data_H1}
\end{figure}

\section{Results on electroweak SUSY production}      
The search for elecctroweak production of SUSY is done with two search regions. 
The VZ signal region is targeting the \firstcharg-\secondchi production decaying to a \PW boson, a \PZ boson and LSP \firstchi. 
This signal region is also sensitive to the production of higgsinos that decay directly to a \firstchi-\firstchi pair, that decay to two \PZ bosons and a gravitino LSP. 
The same production mode is targeted with the ZH signal region, but in this case a the \firstchi is assumed to decay democratically to a \PZ boson and a Higgs boson. 
Good agreement is observed in all signal regions, presented in Table \ref{tab:vzResults} and Table \ref{tab:zhResults} and visualized in Figure \ref{fig:ewkResults}. 
\begin{table}[!hbtp]
\renewcommand{\arraystretch}{1.2}
\setlength{\belowcaptionskip}{6pt}
\small
\centering                             
\caption{\label{tab:vzResults} Predicted and observed event yields are shown for the EW on-\PZ SRs, for each \ptmiss\ bin defined in Table~\ref{tab:WZ}.
The uncertainties shown include both statistical and systematic sources.}
\begin{tabular} {l  c c c c }
\hline\hline
\multicolumn{5}{c}{VZ}\\
\ptmiss [GeV]& 100--150              & 150--250             & 250--350                                      & $>$350 \\ \hline
DY+jets        & 29.3$\pm$4.4         & 2.9$\pm$2.0          & 1.0$\pm$0.7                                   & 0.3$\pm$0.3 \\
FS            & 11.1$\pm$3.6 & 3.2$\pm$1.1  & $0.1^{+0.2}_{-0.1}$                           & $0.1^{+0.2}_{-0.1}$  \\
$\PZ+\nu$         & 14.5$\pm$4.0         & 15.5$\pm$5.1         & 5.0$\pm$1.8                                   & 2.2$\pm$0.9 \\
Total background           & 54.9$\pm$7.0 & 21.6$\pm$5.6 & 6.0$\pm$1.9                           & 2.5$\pm$0.9 \\
Data          & 57                   & 29                   & 2                                             & 0 \\
\hline\hline
\end{tabular}
\end{table}                                                                                                                                                                                     

\begin{table}[!hbtp]
\renewcommand{\arraystretch}{1.2}
\setlength{\belowcaptionskip}{6pt}
\small
\centering                             
\caption{\label{tab:zhResults} Predicted and observed event yields are shown for the EW on-\PZ SRs, for each \ptmiss\ bin defined in Table~\ref{tab:ZH}.
The uncertainties shown include both statistical and systematic sources.}
\begin{tabular} {l  c c c }
\hline\hline
\multicolumn{4}{c}{ZH}\\
\ptmiss [GeV]     & 100--150     & 150--250     & { $>$250 } \\ \hline
DY+jets           & 2.9$\pm$2.4  & 0.3$\pm$0.2  & { 0.1$\pm$0.1 } \\
FS                & 4.0$\pm$1.4  & 4.7$\pm$1.6  & { 0.9$\pm$0.4  } \\
$\PZ+\nu$         & 0.7$\pm$0.2  & 0.6$\pm$0.2  & { 0.3$\pm$0.1 } \\
Total background  & 7.6$\pm$2.8  & 5.6$\pm$1.6  & { 1.3$\pm$0.4 } \\
Data              & 9            & 5            & { 1 } \\ \hline\hline
\end{tabular}
\end{table}                                                                                                                                                                                     


\begin{figure}[htbp!]
\begin{center}
\includegraphics[width=0.45\textwidth]{images/results/Figure_004-a.pdf}
\includegraphics[width=0.45\textwidth]{images/results/Figure_004-b.pdf}
\caption{The \ptmiss distribution is shown for data compared to the background prediction in the on-\PZ VZ (left) and ZH (right) electroweak production SRs.
The lower panel of each figure shows the ratio of observed data to the predicted value in each bin.
The hashed band in the upper panels shows the total uncertainty in the background prediction, including statistical and systematic sources.
The \ptmiss template prediction for each SR is normalized to the first bin of each distribution, and therefore the prediction agrees with the data by construction.}
\label{fig:ewkResults}
\end{center}
\end{figure}      
\section{Results on direct slepton production}    
Finally, the predicted background compared to the observed yields in the signal regions designed to target direct slepton production is presented. 
As opposed to the results previously presented, the signal regions is not only categorized by SF lepton pairs, but is further categorized in dielectron and dimuon pairs. 
The observed number of events in data in the SR are compared with the stacked SM background estimates as shown in Figure \ref{fig:sleptonResults} (SF events), and summarized in Table~\ref{tab:sleptonResults} for SF events and in Table~\ref{tab:resultseemm} for dielectron and dimuon events, separately.
The \mttwo shape of the stacked SM background estimates, the observed data and three signal scenarios are also shown in Figure \ref{fig:sleptonResults}, for SF events, with all SR selection applied except the \mttwo requirement.
\begin{table}[!hbtp]
\renewcommand{\arraystretch}{1.2}
\setlength{\belowcaptionskip}{6pt}
\small                               
\centering
\caption{The predicted SM background contributions, their sum and the observed number of SF events in data.
The uncertainties associated with the background yields stem from statistical and systematic sources.}
\label{tab:results}
\begin{tabular}{c c c c c}
    \hline\hline
    \multicolumn{5}{c}{Same flavor events} \\
    $\ptmiss$ [{\GeVns}] & 100--150 & 150--225 & 225--300 & ${\geq}300$  \\ \hline
    FS bkg. &$96^{+13}_{-12}$ &$15.3^{+5.6}_{-4.5}$ &$4.4^{+3.6}_{-2.3}$ &$1.1^{+2.5}_{-1.0}$\\
    $\PZ\PZ$ &$13.5\pm1.5 $ &$9.78\pm1.19$&$2.84\pm0.56$&$1.86\pm0.12$\\
    $\PW\PZ$ &$6.04\pm1.19$ &$2.69\pm0.88$&$0.86\pm0.45$&$0.21\pm0.20$\\
    DY+jets&$2.01^{+0.39}_{-0.23}$& $0.00+0.28$ &$0.00+0.28$ &$0.00+0.28$\\
    Rare processes&$0.69\pm0.44$&$0.68\pm0.47$&$0.00+0.20$&$0.05\pm0.12$ \\
    Total prediction &$118^{+13}_{-12}$ &$28.4^{+5.9}_{-4.8}$ &$7.9^{+3.7}_{-2.4}$ &$3.2^{+2.6}_{-1.1}$\\
    Data &101 &31 &7 &7\\\hline\hline
\end{tabular}
\end{table}


\begin{table}[!hbtp]
\renewcommand{\arraystretch}{1.2}
\setlength{\belowcaptionskip}{6pt}
\small                               
\centering
\caption{The predicted SM background contributions, their sum and the observed number of dielectron (upper) and dimuon (lower) events in data.
The uncertainties associated with the yields stem from statistical and systematic source.}
\label{tab:resultseemm}
\begin{tabular}{c c c c c}
    \hline\hline
    \multicolumn{5}{c}{Dielectron events} \\
    $\ptmiss$ [{\GeVns}]& 100--150 & 150--225 & 225--300 & ${\geq}300$  \\ \hline
    FS bkg.&$36.1^{+6.6}_{-6.3}$ &$5.7^{+2.5}_{-2.1}$ &$1.6^{+1.5}_{-1.1}$ &$0.41^{+1}_{-0.5}$\\
    $\PZ\PZ$ &$5.17\pm0.68$&$3.79\pm0.58$&$1.18\pm0.31$&$0.69\pm0.07$\\
    $\PW\PZ$ &$2.65\pm0.68$&$1.16\pm0.45$&$0.39\pm0.33$&$0.21\pm0.20$\\
    DY+jets&$0.98^{+0.14}_{-0.15}$&$0.00+0.28$&$0.00+0.28$&$0.00+0.28$\\
    Rare processes&$0.02\pm0.14$&$0.26\pm0.21$&$0.00+0.11$&$0.06\pm0.04$ \\
    Total prediction &$45^{+6.7}_{-6.4}$ &$11.0^{+2.6}_{-2.3}$ &$3.2^{+1.6}_{-1.2}$ &$1.4^{+1.1}_{-0.6}$\\
    Data &45 &10 &2 &2\\
\end{tabular}
\begin{tabular}{c c c c c}
    \hline
    \multicolumn{5}{c}{Dimuon events} \\
    $\ptmiss$ [{\GeVns}]& 100--150& 150--225 & 225--300& ${\geq}300$  \\ \hline
    FS bkg.&$61.3^{+9.1}_{-8.5}$ &$9.8^{+3.9}_{-3.2}$ &$2.8^{+2.4}_{-1.7}$ &$0.70^{+1.7}_{-0.8}$\\
    $\PZ\PZ$ &$8.33\pm0.99$&$5.98\pm0.80$&$1.67\pm0.42$&$1.17\pm0.10$\\
    $\PW\PZ$ &$3.40\pm0.91$&$1.53\pm0.73$&$0.47\pm0.30$&$0.00+0.06$\\
    DY+jets&$1.03^{+0.33}_{-0.14}$ &$0.00+0.28$&$0.00+0.28$&$0.00+0.28$\\
    Rare processes&$0.66\pm0.41$&$0.42\pm0.35$&$0.00+0.16$&$0.00+0.11$ \\
    Total prediction &$75^{+9.2}_{-8.7}$ &$17.7^{+4.1}_{-3.4}$ &$4.8^{+2.5}_{-1.8}$ &$1.9^{+1.7}_{-0.8}$\\
    Data &56 &21 &5 &5\\\hline\hline
\end{tabular}
\end{table}

\begin{figure}[ht!]
\centering
\includegraphics[width=0.45\textwidth]{images/results/Figure_003.pdf}
\includegraphics[width=0.45\textwidth]{images/results/Figure_004.pdf}
\caption{The \ptmiss (left) distribution for the resulting SM background yields estimated in the slepton SF analysis SR overlayed with the observed data as black points.
The \mttwo distribution (right) in the same signal region with three signal scenarios overlayed.}
\label{fig:results}
\end{figure}

