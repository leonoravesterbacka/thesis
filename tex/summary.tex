\chapter{Conclusions}
\noindent\justify
The theoretical motivations behind Supersymmetry (SUSY) as an extension of the Standard Model of particle physics (SM) are numerous. 
The introduction of a new set of SUSY particles, superpartners to the SM particles, has the ability to solve almost all of the outstanding problems with the current formulation of the SM. 
The overwhelming evidence for the existence of non-luminous matter, dark matter, could be explained by SUSY, that predicts a massive particle that does not interact with ordinary matter through any of the known fundamental forces. 
%Moreover, after the discovery of a Higgs boson of 125\GeV in 2012, the SM it is evident that the SM as it is formulated currently exhibits extreme fine-tuning in order to 
\newpara
\noindent\justify
These SUSY particles have the potential of being produced in proton-proton collisions, and indirectly detected by large high energy physics experiments. 
The ATLAS and CMS experiments, at the LHC at CERN, have conducted searches for SUSY particles during the first operational run of the LHC. 
This thesis presents a set of searches for SUSY particles at the unprecedented center-of-mass energy of 13\TeV. 
The strategy deployed in this thesis is to look for events containing two electrons or two muons of opposite electric charge, in association with jets and large missing transverse momentum. 
This particular final state is a powerful tool, as it is sensitive to many types of SUSY particles. 
\newpara
\noindent\justify 
This thesis has presented searches for SUSY particles produced electroweak and strong production modes, originally published in \cite{Sirunyan:2017qaj} and \cite{Sirunyan:2018nwe}. 
The integrated luminosity of the data sample analyzed corresponds to 35.9\fbinv, collected by the CMS detector at a center-of-mass energy of 13\TeV, during the 2016 running of the LHC.
During the Run 1 of the LHC, both CMS and ATLAS reported modest excesses in the strong SUSY production modes.
With the increase in center-of-mass energy and luminosity that the Run 2 offered, the first part of this thesis work consisted of confirming these excesses. 
With the 35.9\fbinv, it became clear that the SM backgrounds model the data very well, thus rejecting the hypothesis that there would be a hint of strongly produced SUSY. 
\newpara
\noindent\justify
The search continued, now on the hunt for electroweak SUSY. 
The superpartners of the SM gauge bosons, the charginos and neutralinos, and the higgsinos, are much lower cross section processes, but have great theoretical features if they existed. 
For example, a light higgsino can solve the hierarchy problem of the SM. 
Two searches were designed, one targeting chargino and neutralino production, and one targeting higgsino production. 
My most significant contribution to the search for electroweak superpartners was the design of the signal region that targeted massive electroweak superpartner by selecting boosted decay products. 
The search for charginos and neutralinos pushed the exclusion of these superpartners by up to 300\GeV, and the search for higgsinos by up to 200\GeV. 
\newpara
\noindent\justify
My final, and most significant, contribution to the quest for finding physics beyond the SM, was done in the context of superpartners of the SM charged leptons. 
The direct production of selectrons and smuons is done at a very low rate at hadron colliders, but has the advantage of resulting in a very clean final state, consisting of only two electrons or muons of opposite charge and large \ptmiss from the neutralino LSP. 
Again, there was no hint of SUSY in the data, but the exclusion limits were pushed by up to 200\GeV with respect to the Run 1 exclusion.  
\newpara
\noindent\justify
At the time of writing this thesis, the LHC Run 2 data taking has just ended. 
The analysis strategies exploited in the searches presented in this thesis will be further developed and incorporated in the future versions of the analysis using the full Run 2 dataset.
Although the results presented in this thesis has provided impressive exclusion limits for a number of SUSY models, there is still immense discovery potential for the lower cross section SUSY models.
The increase in instantaneous luminosity foreseen during the Run 3 of the LHC, and the subsequent High Luminosity LHC, will provide larger data samples which is crucial to probe the rarer SUSY processes. 
This thesis has provided a summary of the performance the \ptmiss reconstruction algorithms during current and future data taking conditions, in the context of pileup.
The algorithm development, especially the PUPPI algorithm, has been shown in this thesis to be a promising algorithm to cope with the increase in pileup and the challenging environment of the HL LHC.
 
