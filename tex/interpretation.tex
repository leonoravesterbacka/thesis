\chapter{Interpretation}\label{sec:interpretation}
In the absence of any significant excess, the results are interpreted in terms of models of new physics. 
As mentioned in the introduction of the thesis, the opposite sign lepton final state is a powerful search tool as it can target many SUSY production modes. 
A short recap of the signal models will be given hereafter, together with an introduction to the statistical analysis that enables limit setting on various SUSY particles.
The sources of the systematic uncertainties for the various searches will be discussed, as they are a crucial ingredient in the limit setting procedure. 
The chapter ends with the exclusion curves of the signal models.

\section{Statistical analysis}
The statistical analysis is outlined in the following section. 
The model parameter estimation and statistical model inference methods are presented along with the means by which the signal is extracted is described. 
For a more detailed discussion of the statistical methods can be found in \cite{Lista:2016chp}. 
\section{Systematic uncertaintites}
Background and signal simulation are subject to effects due to experimental and theoretical sources of uncertatinty.
The interpretation of the results presented in the previous chapter in terms of signal models relies on an appropriate treatment of these systematic uncertainties. 
This section contains a summary of the sources of systematic uncertainties and the final impact they have on the search regions. 
