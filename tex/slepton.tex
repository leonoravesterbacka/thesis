\chapter{Search for direct slepton production}\label{sec:slepton}
\noindent
\justify
Finally, the production mode with the lowest cross section associated to it is treated. 
Direct production of selectrons and smuons provide a very clean final state that makes for a simple but effective search design. 
To first order, no hadronic activity is expected, giving a great handle to reducing SM background processes by imposing a jet veto. 
A 100\% branching ratio of sleptons to leptons and the \firstchi LSP is assumed, that allows for a search with large \ptmiss requirements. 
This chapter summarizes the analysis strategy, the various background prediction techniques and concludes by setting limits on a selectrons and smuons. 
\newpage
\section{Analysis strategy}   
\noindent
\justify
The searches presented so far have the commonality of requiring at least two jets in the final state.
The search for direct slepton production differs in the fact that no hadronic activity is expected.
This results in different sources of SM backgrounds. 
The slepton SR is designed using the leptonic \mttwo variable. 
Further, jets within $|\eta|=2.4$ and of $\pt>25\GeV$ are vetoed.
In the search for sleptons, limits are set on the selectrons and smuon production in dielectron and dimuon final states separately.
The reason for this is because a priori, the mass of the selectrons and smuons have no reason to be the same, in the same way as the mass of the electrons and muons are not the same. 
For this reason, the SR is defined for dielectron and dimuon events separately, and combined.
\begin{table}[ht!]
\def\arraystretch{1.2}
 \caption{Summary of the direct slepton prodcution SR.}
    \label{tab:ZH}
    \begin{center}
    \begin{tabular}{l l l l l }
    \hline \hline
    \multicolumn{5}{c}{Slepton SR}                \\
    Region          & Flavor & $N_{\mathrm{jets}}$  & \mttwo [GeV]& \ptmiss [GeV]\\\hline
    Slepton         & $\MuMu$+$\ElEl$& $=0$         & $\geq90$         & [100, 150, 225, 300+]\\
    Selectron       & $\ElEl$& $=0$                 & $\geq90$         & [100, 150, 225, 300+]\\
    Smuon           & $\MuMu$& $=0$                 & $\geq90$         & [100, 150, 225, 300+]\\
\hline\hline
\end{tabular}       
\end{center}        
\end{table}  

\section{Backgrounds}
\noindent
\justify
\subsection*{Flavor symmetric background}
\noindent
\justify
The main FS background in the slepton search is stemming from \PWW production.
This is in constrast to the general strong and electroweak search, where the main background is due to \ttbar.
The \ttbar is reduced in the slepton search as all jets are vetoed, making the \PWW the main FS contribution.
As the \ttbar enriched control region that is used to measure the \Rsfof in the direct meassurement method is very similar yet orthogonal to the Edge signal region, an extrapolation of this factor to     the signal region is easily validated.
The problem arise in the slepton search, where the \ttbar control region is far from the slepton signal region that has a veto on jets.
For this reason, additional checks are performed to validate the various factors of the FS background prediction method as a function of number of jets.
As can be seen in Figure \ref{fig:rmueSlepton}, no trend is observed in the \rmue and the 0.5($r_{\mu/e}^{corr.}+1/r_{\mu/e}^{corr.}$) variables in the low jet multiplicities, indicating that there is     no problem to extrapolate the results to the slepton SR.
\begin{figure}[htbp!]
\begin{center}
    \includegraphics[width=0.45\textwidth]{images/rsfof/rMuE_ZPeakControl_Run2016_36fb_NJets_corrected.pdf}
    \includegraphics[width=0.45\textwidth]{images/rsfof/rSFOFFromRMuE_ZPeakControl_Run2016_36fb_NJets_corrected.pdf}
    \caption{Dependency of the \rmue (left) and 0.5($r_{\mu/e}^{corr.}+1/r_{\mu/e}^{corr.}$) (right) on the number of jets for data and MC. No significant trend is observed in the lower jet multiplici    ties, indicating no problem when extrapolation the results to the slepton SR.}
\label{fig:rmueSlepton}
\end{center}
\end{figure}
However, a trend is observed for the \Rsfof from the direct measurement in the 0 jet bin. This can be seen in Figure \ref{fig:rsfofSleptonOne}.
\begin{figure}[htbp!]
\begin{center}
    \includegraphics[width=0.45\textwidth]{images/rsfof/plot_rsfof_njetFSonly.png}
    \caption{\Rsfof as a function of number of jets derived in a \ttbar enriched region after relaxing the 2 jet requirement. The MC simulation takes into account all FS processes. In order to not unb    lind the signal region, events with are excluded.}
\label{fig:rsfofSleptonOne}
\end{center}
\end{figure} 
This trend in the first jet bin could indeed cause some worry.
This increase in the number of same flavor events in the 0 jet bin is due to the DY process, which is present in the \ttbar enriched control region after relaxing the jet requirement.
This statement is further motivated by the fact that the discrepancy show up in data and not in MC where only FS processes are included.
But since the signal region is designed with a cut of \mttwo$\geq90\GeV$, the \Rsfof can instead be validated in the jet veto case by requiring a \mttwo$\geq40\GeV$ which is reducing a large fraction of DY contribution while maintaining reasonable statistics.
Figure \ref{fig:rsfofSleptonTwo} shows the \Rsfof as a function of number of jets in the \ttbar control region, but with a requirement that the events that have 0 jets must have a \mttwo within $40-90\GeV$ (the upper requirement is added to avoid unblinding), and a good agreement between data and simulation is obtained, indicating that the extrapolation to the 0 jet SR is fine as long as the SR is defined with a \mttwo requirement.
\begin{figure}[htbp!]
\begin{center}
    \includegraphics[width=0.45\textwidth]{images/rsfof/plot_rsfof_njet.png}
\caption{\Rsfof as a function of number of jets derived in a \ttbar enriched control region after relaxing the 2 jet requirement and requiring that in a 0 jet event, a requirement of the \mttwo to be     within $40-90\GeV$ is imposed. The MC simulation takes into account all flavor symmetric and non-flavor symmetric processes such as DY.}
\label{fig:rsfofSleptonTwo}
\end{center}
\end{figure}
Further, the slepton search is the only search presented in this thesis that has interpretations in the dielectron and dimuon channels separately.
The reason behind this is that there is no a priori reason why the selectrons and smuons should have the same masses, so grouping them together and only interpret in the terms of same-flavor sleptons     is not enough\footnote{Of course, the interpretation in selectrons and smuons separately is also desirable as it gives separate entries in the PDG booklet. }
But since the factorization method utilizes the measurement of \rmue and \RT, that uses a mixture of measurements for electrons and muons.
This means that the \rmue or \RT can not be used to predict the $N_{ee}$ or $N_{\mu\mu}$ separatley.
Instead only the \Reeof and \Rmmof ratios from the direct measurment listed in Table \ref{tab:rSFOF} can be used to predict the $N_{ee}$ or $N_{\mu\mu}$.
The price to pay by only using one of the methods instead of the combined one is that one can not profit from the reduced systematic uncertainty that on gets in the combination of the two metohds.
But, as might have already been noted, the drawback of the FS background prediction method is the large statistical errors associated to the poorly populated OF event bins in the SR.
So a reduction in the systematic uncertainty as a result of a combination of the two methods is still a small reduction compared to the large statistical error.
Table \ref{tab:FlavSymBackgroundsSlepton} summarizes the resulting background estimates for FS backgrounds in the slepton signal regions, split into the SF leptons, dielectron and dimuon signal regions.
\begin{table}[ht!]
\def\arraystretch{1.2}
\setlength{\belowcaptionskip}{6pt}
\small
\centering
\caption{Resulting estimates for flavor-symmetric backgrounds in the Slepton search.}
\label{tab:FlavSymBackgroundsSlepton}
\begin{tabular}{ c  c  c  c  c  c c}
\hline \hline
\ptmiss [GeV] & $N_{OF}$ & $N_{SF}^{factorization}$ & $\Rsfof^{factorization}$ & $\Rsfof^{direct}$  & $\Rsfof^{combined}$ & $N_{SF}^{final}$ \\ \hline
\multicolumn{6}{c}{SF lepton SR} \\\hline
100-150    & 88    & $97^{+12.5}_{-11.5}$  &  $1.08\pm0.07$  &  $1.11\pm0.01$  &  $1.09\pm0.01$  &  $96^{+13}_{-12}$ \\
150-225    & 14    & $15.4^{+5.7}_{-4.5}$  &  $1.08\pm0.07$  &  $1.11\pm0.01$  &  $1.09\pm0.01$  &  $15.3^{+5.6}_{-4.5}$ \\
225-300    & 4     & $4.4^{+3.4}_{-2.4}$   &  $1.08\pm0.07$  &  $1.11\pm0.01$  &  $1.09\pm0.01$  &  $4.4^{+3.6}_{-2.3}$ \\
$>$300     & 1     & $1.1^{+2.6}_{-1.1}$   &  $1.07\pm0.07$  &  $1.11\pm0.01$  &  $1.09\pm0.01$  &  $1.1^{+2.5}_{-1.0}$ \\ \hline
\multicolumn{6}{c}{Dielectron SR} \\\hline
100-150    & 88    & - &  -  &  $0.41\pm0.01$  &  $0.41\pm0.01$  &  $36.1^{+6.6}_{-6.3}    $ \\
150-225    & 14    & - &  -  &  $0.41\pm0.01$  &  $0.41\pm0.01$  &  $5.7^{+2.5}_{-2.1}$ \\
225-300    & 4     & - &  -  &  $0.41\pm0.01$  &  $0.41\pm0.01$  &  $1.6^{+1.5}_{-1.1}$ \\
$>$300     & 1     & - &  -  &  $0.41\pm0.01$  &  $0.41\pm0.01$  &  $0.41^{+1}_{-0.5}$ \\   \hline
\multicolumn{6}{c}{Dimuon SR} \\\hline
100-150    & 88    & -  &  -  &  $0.70\pm0.01$  &  $0.70\pm0.01$  &  $61.3^{+9.1}_{-8.5}$ \\
150-225    & 14    & -  &  -  &  $0.70\pm0.01$  &  $0.70\pm0.01$  &  $9.8^{+3.9}_{-3.2}$ \\
225-300    & 4     & -  &  -  &  $0.70\pm0.01$  &  $0.70\pm0.01$  &  $2.8^{+2.4}_{-1.7}$ \\
$>$300     & 1     & -  &  -  &  $0.70\pm0.01$  &  $0.70\pm0.01$  &  $0.7^{+1.7}_{-0.8}$ \\
\hline\hline
\end{tabular}
\end{table}
\subsection*{\PWZ$\rightarrow3l\nu$ background}
\noindent
\justify
The three lepton control region to target \PWZ in the slepton search is defined as
\begin{itemize}
    \item three tight ID leptons of any flavor or charge with $\pt>25\GeV$ for the leading and $\pt>20\GeV$ for the subsequent leptons
    \item $\ptmiss>70\GeV$
    \item the leptons that form the best \PZ candidate is required to have \mll within 76 to 106\GeV.
    \item $\mathrm{M_{T}}>50\GeV$, where the $\mathrm{M_{T}}$ is constructed with the lepton that does not form the best Z candidate, and is thus assumed to be originating from the W decay.
    \item Jet veto ($\pt>25\GeV$)
\end{itemize}
In Figure~\ref{fig:WZmet}, the \ptmiss and the \mttwo constructed using the leptons from the \PZ boson decay, in the three lepton control region.
The agreement in these signal region variables gives confidence that the \PWZ process is well modelled.
\begin{figure}[htbp!]
\begin{center}
\includegraphics[width=0.45\textwidth]{images/Znu/plot_met_3l_Zmass.pdf}
\includegraphics[width=0.45\textwidth]{images/Znu/Figure_002-b.pdf}
\caption{The \ptmiss (left) and the \mttwo in the three lepton control region in data and MC. }
\label{fig:WZmet}
\end{center}
\end{figure}
From this region, a transfer factor is derived that is applied in the slepton signal region to account for the slight difference in normalization observed in the three lepton control region.
An additional systematic uncertainty is assigned to this method, by letting the jet energy scale corrections vary up and down in the computation of the \ptmiss, and the difference in the yield in the     background and \PWZ MC is propagated to the error on the final scale factor.
The resulting transfer factor is $1.06\pm0.06$, where the 6\% is the statistical uncertainty on the data in the control region which is taken as a systematic uncertainty on the method.
Further, systmeatic uncertainties such as the variations of the jet energy scale and resolution, as well as pdf anc scale variations is evaluated for this method, which will be further explained in Se    ction\ref{sec:interpretations}.
The transfer factor is presented in Table \ref{tab:WZslepton}, using the MC samples summarized in Table\ref{tab:3lMCsamples} in Appendix A.
\begin{table}[ht!]
\def\arraystretch{1.2}
\setlength{\belowcaptionskip}{6pt}
\small
\centering
\caption{Transfer factor derived in the three lepton control region for slepton search.}
\label{tab:WZslepton}
\begin{tabular}{l c }
\hline \hline
\multicolumn{2}{c}{Slepton 3 lepton CR}  \\\hline
signal MC        & 368.46     $\pm$  6.01    \\
bkg. MC          & 52.93  $\pm$  4.90\\ \hline
\textbf{data}       & \textbf{445}  \\
data-bkg.        &  392.07   $\pm$  21.66 \\ \hline
(data-bkg.)/sig. & 1.06   $\pm$  0.06\\                  \hline\hline
\end{tabular}
\end{table}
\subsection*{\PZZ$\rightarrow2l2\nu$ background}
\noindent
\justify
In the slepton search, the \PZZ$\rightarrow2l2\nu$ process is dominant, especially in the high \ptmiss signal regions.
This process can enter the signal region if the leptonically decaying \PZ boson is off-shell, producing a lepton pair that escapes the \PZ boson veto.
This process is estimated from simulation, after validating the simulation in a 4 lepton control region.  
This control region is enriched in the Z$\mathrm{Z^{*}}\rightarrow$4l process.
This control region enables for the validation of the modelling of the lepton pair from the off-shell \PZ boson, by treating the other two leptons that are compatible with a good \PZ candidate as neutrinos. 
The \pt of the leptons forming the best \PZ candidate is added to the \ptmiss, so to mimic the \ptmiss produced if these charged leptons were replaced with neutrinos.
The agreement betweent the data and the simulation is checked in variables such as the \ptmiss and the \mttwo recomputed with the remaining leptons and the emulated \ptmiss.
The four lepton control region used for the slepton search is defined as
\begin{itemize} 
    \item four tight ID leptons of any flavor or charge with \pt $>$ 25 GeV for the leading and \pt $>$ 20 GeV for the subsequent leptons
    \item the leptons that form the best Z candidate is required to have \mll within 76 to 106 GeV. 
    \item the leptons that form the other Z candidate is required to have \mll within 50 to 130 GeV.
    \item Jet veto (\pt $>$ 25 GeV) 
\end{itemize}                       
A NNLO k-factor is applied to the $ZZ\rightarrow 2l2\nu$ and $ZZ\rightarrow 4l$ simulation, which is provided as a function of the generator level \pt, mass and \dphi of the diboson system.
The diboson \pt dependent k-factor is eventually applied to the ZZTo2L2Nu prediction in the signal region, and an uncertainty on the shape is derived as the difference between the distributions after     applying the \pt dependent k-factor and no k-factor, after both distributions are normalized to their areas.
The distributions of the generator \pt and mass of the diboson systems in the slepton signal region is presented in Figure~\ref{kfactor}, after the respective k-factor has been applied.
\begin{figure}[htbp!]
\begin{center}
\includegraphics[width=0.45\textwidth]{images/Znu/plot_GENptZZ_kFactor.pdf}
\includegraphics[width=0.45\textwidth]{images/Znu/plot_GENmassZZ_kFactor.pdf}
\caption{Distributions of the generator \pt, mass and \dphi of the diboson system, in the signal region, with and without the respective k-factor applied, in ZZTo2L2Nu MC.}
\label{kfactor}
\end{center}
\end{figure}
Further information on the k-factors is documented in Appendix A.
A four lepton control region is constructed, which will be used to estimate the $ZZ\rightarrow 2l2\nu$ background. The four lepton signature is very pure and provides a good data/MC agreement.
Figure~\ref{fig:ZZslepton} shows the invariant mass of the best Z candidate and the \mttwo, in data and MC.
\begin{figure}[htbp!]
\begin{center}
\includegraphics[width=0.45\textwidth]{images/Znu/plot_bestMll_4l.pdf}
\includegraphics[width=0.45\textwidth]{images/Znu/Figure_002-a.pdf}
\caption{The invariant mass of the two lepton pair forming the best \PZ candidate in the four lepton control region (left). The emulated \mttwo is the \mttwo formed with the \ptmiss that has the \pt o    f the leptons forming the best \PZ candidate added to it, and the remaining two leptons (right).}
\label{fig:ZZslepton}
\end{center}
\end{figure}
In Table \ref{tab:tab4lcontrol} the transfer factor of the four lepton control region is summarized, where the signal simulation is the ZZ $\rightarrow$4l process summarized in Table \ref{tab:4lMCsamp    les} of Appendix A (here all production modes are included as opposed to in the on-Z search where many of the samples were not generated at the time of the publication).
A transfer factor of 0.94 is obtained with a statistical uncertainty of 7\%, and this 7\% is used as a systematic uncertainty on the method. 
As the slepton signal region is defined with a selection of leading leptons of \pt greater than 50 \GeV, this differs from the control region definition of \pt greater than 25 \GeV to maintain the sta    tistical power of the sample. 
\begin{table}[ht!]
\def\arraystretch{1.2}
\setlength{\belowcaptionskip}{6pt}
\small                
\centering                        
\caption{Transfer factor derived in the four lepton control region for slepton search.}
\label{tab:tab4lcontrol}          
\begin{tabular}{l c }
\hline \hline
\multicolumn{2}{c}{Slepton 4 lepton CR}  \\\hline
signal MC        & 184.7     $\pm$  1.2    \\
bkg. MC          & 5.7  $\pm$  1.7\\ \hline
\textbf{data}       & \textbf{179}  \\
data-bkg.        &  173.4   $\pm$  13.5 \\ \hline
(data-bkg.)/sig. & 0.94   $\pm$  0.07\\\hline\hline
\end{tabular}    
\end{table}
\subsection*{Drell-Yan}
\noindent
\justify
The Drell-Yan is a subdominant background in the slepton search, as it is heavily suppressed by the \PZ boson veto, the jet veto, the large \mttwo and \ptmiss requirements.
For this reason the background is predicted from simulation, in order to avoid large systematic uncertainties due to the poorly poulated \ptmiss bins.
Similarly to the Drell-Yan treatment in the edge search, an \Routin method is used to transfer the prediction of events on the \PZ peak to outside of the \PZ window.
The difference is that the events in the \PZ window is taken from simulation.
As the slepton signal region is not binned in the invariant mass like the edge search is, the \Routin only needs to be computed twice, \Routin above and below the \PZ window.
\begin{table}[ht!]
\def\arraystretch{1.2}
\setlength{\belowcaptionskip}{6pt}
\small
\centering
\caption{ Measured values for \Routin for data and MC in the different signal regions of the slepton search.}
\label{tab:rinout}
\begin{tabular}{c c c c}
\hline \hline
\mll [GeV] & $N_{\mathrm{in}}$ & $N_{\mathrm{out}}$ & \Routin  \\
\hline
20-76 & 9096 $\pm$ 96 & 634 $\pm$ 26  & 0.069 $\pm$ 0.05 \\
106+  & 9096 $\pm$ 96 & 628 $\pm$ 27  & 0.069 $\pm$ 0.05 \\\hline\hline
\end{tabular}
\end{table}
\section{Systematic uncertainties}
\noindent
\justify
\subsection*{Systematic uncertainties for diboson backgrounds}
\noindent
\justify
The SM contribution from \PWZ and \PZZ processes is taken directly from MC, corrected with a transfer factor and with additional systematic uncertainties propagated, as described in Sections \ref{sec:ZZ} and \ref{sec:WZ}.
A summary of the sources and the magnitude of the uncertainties considered for \PZZ and \PWZ processes is shown in Table \ref{tab:systematicsZZ} and Table \ref{tab:systematicsWZ}.
For the \PZZ an additional uncertainty, steming from the application of the k-factor, is propagated as the difference in shape observed between applying the generator diboson \pt k-factor and not applying the k-factor.
\begin{table}[!hbtp]
\renewcommand{\arraystretch}{1.2}
\setlength{\belowcaptionskip}{6pt}
\small
\centering
\caption{
List of systematic uncertainties taken into account for the ZZ background prediction method.}
\begin{tabular}{l c c}
\hline\hline
Source of uncertainty                         & Uncertainty (\%)     \\
\hline
Lepton reconstruction and isolation     &      {5}                   \\ %\hline    
Trigger modeling                        &      {3}                   \\ %\hline 
Jet energy scale                        &      {0.5-2}                  \\ %\hline
Renormalization/factorization scales    &      {2-4}                 \\ %\hline  
PDF                                     &      {3-17}                 \\ %\hline  
k-factor shape                          &      {1-3}                 \\ %\hline  
ZZ CR transfer factor             &      {7}                 \\ %\hline  
\hline\hline
\label{tab:systematicsZZ}
\end{tabular}
\end{table}
\begin{table}[!hbtp]
\renewcommand{\arraystretch}{1.2}
\setlength{\belowcaptionskip}{6pt}
\small
\centering
\caption{List of systematic uncertainties taken into account for the WZ background prediction method.}
\begin{tabular}{l c c}
\hline\hline
Source of uncertainty                         & Uncertainty (\%)     \\
\hline
Lepton reconstruction and isolation     &      {5}                   \\ %\hline    
Trigger modeling                        &      {3}                   \\ %\hline 
Jet energy scale                        &      {0.5-6}                  \\ %\hline
Renormalization/factorization scales    &      {3-7}                 \\ %\hline  
pdf                                     &      {0.5-11}                 \\ %\hline  
WZ CR transfer factor                   &      {6}                 \\ %\hline  
\hline\hline
\label{tab:systematicsWZ}
\end{tabular}
\end{table}


\subsection*{Systematic uncertainties for signal}
\noindent
\justify
 The magnitude of the systematic uncertainties in search for direct slepton production is summarizes in Table \ref{tab:systematicsSleptons}.
\begin{table}[!hbtp]
\renewcommand{\arraystretch}{1.2}
\setlength{\belowcaptionskip}{6pt}
\small
\centering
\caption{\label{tab:systematicsSleptons} List of systematic uncertainties taken into account for the signal yields.}
\begin{tabular}{l c}
\hline\hline
Source of uncertainty                   & Uncertainty (\%)\\
\hline
Integrated luminosity                   &      2.5      \\
Lepton reconstruction and isolation     &      5        \\
Fast simulation electron efficiency     &      1--2.5   \\
Fast simulation muon efficiency         &      1--3     \\
Trigger modeling                        &      3        \\
Jet energy scale                        &      1--15    \\
Pileup                                  &      0.5--7   \\
Fast simulation \ptmiss modeling        &      0.5--20  \\
Unclustered energy shifted \ptmiss      &      0.5--8   \\
Muon energy scale shifted \ptmiss       &      0.5--20  \\
Electron energy scale shifted \ptmiss   &      0.5--4   \\
Renormalization/factorization scales    &      1--11    \\
PDF                                     &      3        \\
MC statistical uncertainty              &      0.5--20  \\ \hline\hline
\end{tabular}
\end{table}

\section{Results}    
\noindent
\justify
Finally, the predicted background compared to the observed yields in the signal regions designed to target direct slepton production is presented. 
As opposed to the results previously presented, the signal regions is not only categorized by SF lepton pairs, but is further categorized in dielectron and dimuon pairs. 
The observed number of events in data in the SR are compared with the stacked SM background estimates as shown in Figure \ref{fig:sleptonResults} (SF events), and summarized in Table \ref{tab:sleptonResults} for SF events and in Table \ref{tab:resultseemm} for dielectron and dimuon events, separately.
The \mttwo shape of the stacked SM background estimates, the observed data and three signal scenarios are also shown in Figure \ref{fig:sleptonResults}, for SF events, with all SR selection applied except the \mttwo requirement.
\begin{table}[!hbtp]
\renewcommand{\arraystretch}{1.2}
\setlength{\belowcaptionskip}{6pt}
\small                               
\centering
\caption{The predicted SM background contributions, their sum and the observed number of SF events in data.
The uncertainties associated with the background yields stem from statistical and systematic sources.}
\label{tab:results}
\begin{tabular}{c c c c c}
    \hline\hline
    \multicolumn{5}{c}{Same flavor events} \\
    $\ptmiss$ [{\GeVns}] & 100--150 & 150--225 & 225--300 & ${\geq}300$  \\ \hline
    FS bkg. &$96^{+13}_{-12}$ &$15.3^{+5.6}_{-4.5}$ &$4.4^{+3.6}_{-2.3}$ &$1.1^{+2.5}_{-1.0}$\\
    $\PZ\PZ$ &$13.5\pm1.5 $ &$9.78\pm1.19$&$2.84\pm0.56$&$1.86\pm0.12$\\
    $\PW\PZ$ &$6.04\pm1.19$ &$2.69\pm0.88$&$0.86\pm0.45$&$0.21\pm0.20$\\
    DY+jets&$2.01^{+0.39}_{-0.23}$& $0.00+0.28$ &$0.00+0.28$ &$0.00+0.28$\\
    Rare processes&$0.69\pm0.44$&$0.68\pm0.47$&$0.00+0.20$&$0.05\pm0.12$ \\
    Total prediction &$118^{+13}_{-12}$ &$28.4^{+5.9}_{-4.8}$ &$7.9^{+3.7}_{-2.4}$ &$3.2^{+2.6}_{-1.1}$\\
    Data &101 &31 &7 &7\\\hline\hline
\end{tabular}
\end{table}
\begin{table}[!hbtp]
\renewcommand{\arraystretch}{1.2}
\setlength{\belowcaptionskip}{6pt}
\small                               
\centering
\caption{The predicted SM background contributions, their sum and the observed number of dielectron (upper) and dimuon (lower) events in data.
The uncertainties associated with the yields stem from statistical and systematic source.}
\label{tab:resultseemm}
\begin{tabular}{c c c c c}
    \hline\hline
    \multicolumn{5}{c}{Dielectron events} \\
    $\ptmiss$ [{\GeVns}]& 100--150 & 150--225 & 225--300 & ${\geq}300$  \\ \hline
    FS bkg.&$36.1^{+6.6}_{-6.3}$ &$5.7^{+2.5}_{-2.1}$ &$1.6^{+1.5}_{-1.1}$ &$0.41^{+1}_{-0.5}$\\
    $\PZ\PZ$ &$5.17\pm0.68$&$3.79\pm0.58$&$1.18\pm0.31$&$0.69\pm0.07$\\
    $\PW\PZ$ &$2.65\pm0.68$&$1.16\pm0.45$&$0.39\pm0.33$&$0.21\pm0.20$\\
    DY+jets&$0.98^{+0.14}_{-0.15}$&$0.00+0.28$&$0.00+0.28$&$0.00+0.28$\\
    Rare processes&$0.02\pm0.14$&$0.26\pm0.21$&$0.00+0.11$&$0.06\pm0.04$ \\
    Total prediction &$45^{+6.7}_{-6.4}$ &$11.0^{+2.6}_{-2.3}$ &$3.2^{+1.6}_{-1.2}$ &$1.4^{+1.1}_{-0.6}$\\
    Data &45 &10 &2 &2\\
\end{tabular}
\begin{tabular}{c c c c c}
    \hline
    \multicolumn{5}{c}{Dimuon events} \\
    $\ptmiss$ [{\GeVns}]& 100--150& 150--225 & 225--300& ${\geq}300$  \\ \hline
    FS bkg.&$61.3^{+9.1}_{-8.5}$ &$9.8^{+3.9}_{-3.2}$ &$2.8^{+2.4}_{-1.7}$ &$0.70^{+1.7}_{-0.8}$\\
    $\PZ\PZ$ &$8.33\pm0.99$&$5.98\pm0.80$&$1.67\pm0.42$&$1.17\pm0.10$\\
    $\PW\PZ$ &$3.40\pm0.91$&$1.53\pm0.73$&$0.47\pm0.30$&$0.00+0.06$\\
    DY+jets&$1.03^{+0.33}_{-0.14}$ &$0.00+0.28$&$0.00+0.28$&$0.00+0.28$\\
    Rare processes&$0.66\pm0.41$&$0.42\pm0.35$&$0.00+0.16$&$0.00+0.11$ \\
    Total prediction &$75^{+9.2}_{-8.7}$ &$17.7^{+4.1}_{-3.4}$ &$4.8^{+2.5}_{-1.8}$ &$1.9^{+1.7}_{-0.8}$\\
    Data &56 &21 &5 &5\\\hline\hline
\end{tabular}
\end{table}

\begin{figure}[ht!]
\centering
\includegraphics[width=0.45\textwidth]{images/results/Figure_003.pdf}
\includegraphics[width=0.45\textwidth]{images/results/Figure_004.pdf}
\caption{The \ptmiss (left) distribution for the resulting SM background yields estimated in the slepton SF analysis SR overlayed with the observed data as black points.
The \mttwo distribution (right) in the same signal region with three signal scenarios overlayed.}
\label{fig:results}
\end{figure}
\section{Interpretation}
\noindent
\justify
Upper limits on the direct slepton pair production cross section are displayed in Figure \ref{fig:TOT} for three scenarios: assuming the existence of both flavour mass degenerate left- and
right-handed sleptons, for only left-handed sleptons, and for only right-handed sleptons. 
Similarly, the limits on direct selectron and smuon production are displayed in Figures \ref{fig:TOTee} and \ref{fig:TOTmm}, respectively.
The Figures \ref{fig:TOT} \ref{fig:TOTmm} also show the $95\%$ \CL exclusion contours, as a function of the \slep and \lsp masses.
Note that the cross section at a given mass for right-handed sleptons is expected to be about one third of that for left-handed sleptons.
The analysis probes slepton masses up to approximately 450, 400, or 290\GeV, assuming both left- and right-handed, left-handed only, or right-handed sleptons, and a massless LSP.
For models with high slepton masses and light LSPs the sensitivity is driven by the highest \ptmiss bin. The sensitivity is reduced at higher LSP masses due to the effect of the lepton acceptance.
In the case of selectrons (smuons), the limits corresponding to these 3 scenarios are 350, 310 and 250\GeV (310, 280, and 210\GeV).
Since the dimuon data yield in the highest \ptmiss\ bin is somewhat higher than predicted, the observed limits in this channel are weaker than expected in the absence of signal.
These results improve the previous 8\TeV exclusion limits by 100--150\GeV in the slepton mass~\cite{2012ewk}.
\begin{figure}[htbp]
\centering
\includegraphics[width=0.48\textwidth]{images/interpretation/slepton/Figure_005-a.pdf} \\
\includegraphics[width=0.48\textwidth]{images/interpretation/slepton/Figure_005-b.pdf}
\includegraphics[width=0.48\textwidth]{images/interpretation/slepton/Figure_005-c.pdf}
\caption{\label{fig:TOT}
Cross section upper limit and exclusion contours at 95\% \CL for direct slepton production of two flavours, selectrons and smuons,
as a function of the \lsp and \slep masses, assuming the production of both left- and right-handed sleptons
(upper) or production of only left- (lower left) or right-handed (lower right).
The region under the thick red dotted (black solid) line is excluded by the expected (observed) limit.
The thin red dotted curves indicate the regions containing 95\% of the distribution of limits
expected under the background-only hypothesis.
The thin solid black curves show the change in the observed limit due to
variation of the signal cross sections within their theoretical uncertainties.
}
\end{figure}
\begin{figure}[htbp]
\centering
\includegraphics[width=0.48\textwidth]{images/interpretation/slepton/Figure_006-a.pdf} \\
\includegraphics[width=0.48\textwidth]{images/interpretation/slepton/Figure_006-b.pdf}
\includegraphics[width=0.48\textwidth]{images/interpretation/slepton/Figure_006-c.pdf}
\caption{\label{fig:TOTee}
Cross section upper limit and exclusion contours at 95\% \CL for direct selectron production
as a function of the \lsp and \slep masses, assuming the production of both left- and right-handed selectrons
(upper), or production of only left- (lower left) or right-handed (lower right) selectrons.
The region under the thick red dotted (black solid) line is excluded by the expected (observed) limit.
The thin red dotted curves indicate the regions containing 95\% of the distribution of limits
expected under the background-only hypothesis. For the right-handed selectrons, only the $+1\sigma$ expected line (thin red dotted curve) is shown as no exclusion can be made at $-1\sigma$.
The thin solid black curves show the change in the observed limit due to
variation of the signal cross sections within their theoretical uncertainties.
}
\end{figure}
 \begin{figure}[htbp]
\centering
\includegraphics[width=0.48\textwidth]{images/interpretation/slepton/Figure_007-a.pdf} \\
\includegraphics[width=0.48\textwidth]{images/interpretation/slepton/Figure_007-b.pdf}
\includegraphics[width=0.48\textwidth]{images/interpretation/slepton/Figure_007-c.pdf}
\caption{\label{fig:TOTmm}
Cross section upper limit and exclusion contours at 95\% \CL for direct smuon production
as a function of the \lsp and \slep masses, assuming the production of both left- and right-handed smuons
(upper), or production of only left- (lower left) or right-handed (lower right) smuons.
The region under the thick red dotted (black solid) line is excluded by the expected (observed) limit.
The thin red dotted curves indicate the regions containing 95\% of the distribution of limits
expected under the background-only hypothesis. For the right-handed smuons, only the $+1\sigma$ expected line (thin red dotted curve) is shown as no exclusion can be made at $-1\sigma$.
The thin solid black curves show the change in the observed limit due to
variation of the signal cross sections within their theoretical uncertainties.
}
\end{figure}
\section{Summary}
\noindent
\justify
A search for direct slepton (selectron or smuon) production, in events with opposite-sign, same-flavour leptons, no jets, and missing transverse momentum has been presented.
The data comprise a sample of proton-proton collisions collected with the CMS detector in 2016 at a centre-of-mass energy of 13\TeV, corresponding to an integrated luminosity of \lint.
Observations are in agreement with expectations from the SM within the statistical and systematic uncertainties.
Exclusion limits are provided assuming right-handed only, left-handed only and right-and left-handed two flavour slepton production scenarios (mass degenerate selectrons and smuons).
Slepton masses up to 290, 400 and 450\GeV respectively are excluded at 95\% confidence level, assuming a massless LSP. 
Exclusion limits are also provided assuming a massless LSP and right-handed only, left-handed only and right-and left-handed single flavour production scenarios, excluding selectron (smuon) masses up to 250, 310 and 350\GeV (210, 280 and 310\GeV), respectively.
These results improve the previous exclusion limits measured by the CMS experiment at a centre-of-mass energy of 8\TeV by 100–-150\GeV in slepton masses.
