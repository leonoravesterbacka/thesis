\chapter{SUPERSYMMETRY} \label{sec:susy}
\noindent\justify
As is clear from the previous chapter, there is a need for a theory that can provide answers to the open questions of the SM. 
Many attempts to formulate theories beyond the SM have been made, such as theories involving extra hidden dimensions \cite{ArkaniHamed:1998rs}, technicolor \cite{Hill:2002ap} and grand unified theories with an $SU(5)$ invariance \cite{Georgi:1974sy}.
As providing a description of all the BSM theories on the market is out of the scope of this thesis, emphasis will be put on the most popular extension of the SM, namely $Supersymmetry$ (SUSY) \cite{Wess:1973kz, Fayet:1974pd, Nilles:1983ge}. 
SUSY was formulated in the 1970s and is able to tackle many of the problems listed at the end of the previous chapter. 
Similar to the formulation of the SM, that introduces particles with partners of opposite electric charge or chirality, SUSY adds a final symmetry to the theory. 
This final symmetry is between fermions and bosons, and dictates that there exists a fermionic SUSY partner to each boson and vice versa. 
\newpara
\noindent\justify
This chapter contains a review of the known problems of the SM, and how SUSY could solve them, closely following the excellent overview given in \cite{Martin:1997ns}. 
An introduction to the so called Minimal Supersymmetric Standard Model (MSSM) is provided in Section \ref{sec:mssm}. 
The formulation of the MSSM Lagrangian is given, along with a description of the procedure by which SUSY breaks to give masses of the SUSY particles different to that of their SM partners. 
\newpage
\section{Problems of the SM}
\noindent\justify
As summarized at the end of the last chapter, the SM is far from a complete theory as it is unable to provide explanations of phenomena such as the origin of dark matter and neutrino masses. 
The fact that it also relies on many parameters and requires extreme fine tuning in order to hold motivates the introduction of a more aesthetically pleasing extension of the model. 
The hierarchy problem of the SM and the problem with the non-converging coupling constants of the SM are introduced in the following. 
\subsection*{The hierarchy problem}
\noindent\justify
The so called hierarchy problem, that was introduced at the end of the last chapter, requires fine tuning in order to be able to stabilize the Higgs boson mass \cite{Martin:1997ns}. 
Recalling the problem, the $m_{H}^{2}$ receives enormous quantum corrections that arise from virtual loops of all the particles that couple to the Higgs field. 
If the Higgs field couples to a fermion $f$, corresponding to a term $y_{f}H\bar{f}f$ in the Lagrangian, and there are no chiral or local gauge symmetries, the correction to $m_{H}^{2}$ will be of the form
\begin{equation}
\delta m_{H,f}^{2}=\frac{(y_{f})^{2}}{8\pi^{2}}\left[-\Lambda^{2} +6m_{f}^{2}\ln\frac{\Lambda}{m_{f}}\right]
\label{eq:loop}
\end{equation}                                                                                                
This correction can be visualized in Figure \ref{fig:loop}.
$\Lambda$ is an arbitrary scale that can be interpreted as the energy scale at which new physics phenomena appear. 
The $y_{f}$, first introduced in Equation \ref{eq:yf} is the strength at which the fermion couples to the Higgs field. 
As $y_{f}$ is proportional to the fermion mass, the top quark will contribute most to Equation \ref{eq:loop}.
\begin{figure}[htbp!]
\begin{center}
    \includegraphics[width=0.49\textwidth]{images/theory/loop.jpg}
\caption{One-loop quantum corrections to the Higgs squared mass parameter $m_H^2$, due to a Dirac fermion $f$ (left), and a scalar $S$ (right).}
\label{fig:loop}
\end{center}
\end{figure}                                     
The quadratic divergences introduced in Equation \ref{eq:loop} through the scale $\Lambda$ are not desirable. 
They imply that there is extreme fine tuning at work that tame the divergences, spanning many orders of magnitude.
Another, more aesthetically pleasing, solution to this problem is outlined below.    
The divergences can be canceled through the introduction of a new set of particles, or more precisely, a new bosonic version of each fermion. 
Assume the existence of a scalar version of a SM fermion, of mass $m_{s}$, that couples to the Higgs field with a Lagrangian term of $-\lambda_{s}|H|^{2}|S|^{2}$. 
The corrections to the Higgs mass due to this new boson is of the form
\begin{equation}
\delta m_{H,s}^{2}=\frac{\lambda_{s}}{16\pi^{2}}\left[\Lambda^{2}-2m_{s}^{2}\ln\frac{\Lambda}{m_{s}}\right].
\end{equation}                                                                                                                 
and visualized in the right of Figure \ref{fig:loop}. 
As the correction factors to the Higgs mass are added up, and they are of opposite sign, these two terms will now neatly cancel the divergences, if the couplings are related according to $\lambda_{s}=\lambda_{f}^{2}$. 
The remaining contribution to the Higgs mass is then \cite{Pape:2006ar}:
\begin{equation}
\delta m_{H,s+f}^{2}\approx\frac{\lambda_{f}^{2}}{4\pi^{2}}(m_{s}^{2}-m_{f}^{2})\ln\frac{\Lambda}{m_{s}}
\end{equation}                                                                                                                 
As long as the masses of the new scalar bosons are not too far from that of their SM partners, the corrections to the Higgs mass are well-behaved.
\subsection*{Unification of forces}
\noindent\justify
The coupling constants of the symmetry group of the SM $SU(3)_{C}\otimes SU(2)_{L}\otimes U(1)_{Y}$, $\alpha_{1}$, $\alpha_{2}$ and $\alpha_{3}$ were first introduced in Section \ref{sec:alpha}. 
From their definitions, it is clear that they depend on the energy scale $q^{2}$, and that their values differ from each other. 
Instead of having three distinct coupling constants, it is theorized that these couplings can instead converge to the same value at some energy scale, thus unifying the forces such that $\alpha_{1}=\alpha_{2}=\alpha_{3}=\alpha_{\mathrm{GUT}}$ where the GUT stands for Grand Unified Theory. 
If one were to calculate the evolution of the couplings using renormalization group equations with two-loop effects, the coupling constants evolve as shown by dashed lines in Figure \ref{fig:unification} \cite{Martin:1997ns}.
Here the $\alpha^{-1}$ of the three couplings indeed meet each other but at different energy scales, thus not resulting in the desirable unification of all three forces.   
\newpara
\noindent\justify
As has already been hinted in the beginning of this chapter, SUSY introduces a new set of particles, thus modifying the slopes of the inverse of the coupling constants in Figure \ref{fig:unification}.
By requiring that the three forces should meet, one can calculate at what energy scale that should be, and also the mass scale of the SUSY particles. 
This is visualized in Figure \ref{fig:unification} by solid lines, where the scale at which the forces meet is the so called $m_{\mathrm{GUT}}=10^{16\pm 0.3}\GeV$ yielding a value of $\alpha_{\mathrm{GUT}}^{-1}=25.7\pm1.7$ \cite{Amaldi:1991cn}. 
\begin{figure}[htbp!]
\begin{center}
    \includegraphics[width=0.49\textwidth]{images/theory/unification.png}
    \caption{Two-loop renormalization group evolution of the inverse gauge couplings $\alpha^{-1}$ in the SM (dashed lines) and the MSSM (solid lines). 
In the MSSM case, the sparticle masses are treated as a common threshold varied between 750 GeV and 2.5 TeV, and $\alpha_{3}(m_{\PZ})$ is varied between 0.117 and 0.120. \cite{Martin:1997ns}}
\label{fig:unification}
\end{center}
\end{figure}                                                                         
This unification of forces that would arise in SUSY is indeed an attractive feature. 
\section{Minimal Supersymmetric Standard Model}\label{sec:mssm}
\noindent\justify
In SUSY, each SM particle is associated with a supersymmetric partner, called a superpartner.
Just like how the anti-particle of the electron is the positron that carries opposite electric charge, the superpartners of the SM particles carries spin that differ by $\frac{1}{2}$. 
Formally, this can be described by an operator $Q$ that changes a SM bosonic state $\ket{\mathrm{boson}}$ to a fermionic state $\ket{\mathrm{fermion}}$, and vice versa. 
\begin{equation}
Q\ket{\mathrm{fermion}}=Q\ket{\mathrm{boson}}  ,\,\,\,Q\ket{\mathrm{boson}}=Q\ket{\mathrm{fermion}}
\end{equation}
The operator $Q$ and its hermitian conjugate $Q^{\dagger}$ are fermionic operators (they carry spin $\frac{1}{2}$), implying that SUSY is a spacetime symmetry rather than an internal or gauge symmetry. 
Further, the operators need to satisfy the following commutation and anti-commutation relations
\begin{align}
&\{Q,Q^{\dagger}\}=P^{\mu}\\
&\{Q,Q\}=\{Q^{\dagger},Q^{\dagger}\}=0\\
&[P^{\mu},Q]=[P^{\mu},Q^{\dagger}]=0
\end{align} 
with the four-momentum generator of spacetime translations $P^{\mu}$, which is a bosonic operator \cite{Martin:1997ns}.
$Q$ and $Q^{\dagger}$ form a supersymmetry algebra, $superalgebra$, and the fermions and bosons are arranged in the irreducible representations of this algebra. 
The representations are referred to as $supermultiplets$, and contain fermions and bosons, where the boson is the so called $superpartner$ of the fermion in that supermultiplet. 
The multiplets contain an equal number of fermions and bosons, $n_{\mathrm{fermion}} = n_{\mathrm{boson}}$. 
Further, particles in the same supermultiplet must have the same gauge interaction, meaning that they would carry the same quantum numbers such as color charge, electric charge and the same value of the weak isospin coupling. 
\subsection*{Particle content of the MSSM}
\noindent\justify
The simplest implementation of SUSY is the MSSM, when the number of copies of the operators $Q$ and $Q^{\dagger}$ is 1.\footnote{In general, there could be a multiple of copies of the operators $Q$ and $Q^{\dagger}$, up to 8 \cite{Haag:1974qh,Coleman:1967ad}.}
The immediate implication is that the number of particles is doubled, by including a superpartner of each SM particle, with the exact same quantum numbers except the spin. 
The superpartners of the SM fermions are scalars (spin-0) and are denoted $scalar$-fermions or $sfermions$ for short. 
The superpartners of the SM charged leptons are scalar leptons, $sleptons$, and denoted $selectrons$, $smuons$ and $staus$ respectively. 
Similarly, the superpartners of the quarks are called scalar-quarks, $squarks$. 
In the SM, only left-handed neutrinos exist, hence only superpartners to these left-handed neutrinos, $sneutrinos$ are predicted. 
The superpartners of the SM gauge bosons have a postfix of $-ino$ on the SM boson name. The superpartner of the SM gluon is the glu-$ino$. 
The SM particle symbols are denoted with a tilde in order to specify that they are the superpartner ($g\leftrightarrow\tilde{g}$). 
The SM fermions, their superpartners of the first generation, the SM gauge bosons and their superpartners are listed in Table \ref{tab:susy}.  
The defining feature of supersymmetric partners is that all other quantum numbers other than the spin are identical. 
%This feature is represented by the last column in Table \ref{tab:susy}, where the $(SU(3)_{C},SU(2)_{L},U(1)_{Y})$ charges for a SM particle and its superpartners are listed. 
\begin{table}[ht!]
\def\arraystretch{1.2}
\setlength{\belowcaptionskip}{6pt}
\small
\centering
\caption{An overview of the superpartners of the SM bosons and fermions \cite{Martin:1997ns}.}
\label{tab:susy}
\begin{tabular}{l c c c c }
        \hline \hline
        SM particle \& SUSY partner              & chirality         &spin-0                 & spin-$\frac{1}{2}$                              &spin-1 \\\hline
        \multirow{2}{*}{leptons \& sleptons}     & $L$               &($\slepL,\sneuL$)      & ($\nu_{L}$,$\lepL$)                             &$-$    \\
                                                 & $R$               &$\slepR$               & $\lepR$                                         &$-$    \\\hline
        \multirow{3}{*}{quarks \& squarks}       & $L$               &($\supL,\sdownL$)      & ($\upL$,$\downL$)                               &$-$    \\
                                                 & \multirow{2}{*}{R}&$\supR$                & $\upR$                                          &$-$    \\
                                                 &                   &$\sdownR$              & $\downR$                                        &$-$    \\\hline
        \multirow{3}{*}{gauge bosons \& gauginos}& $-$               &$-$                    & $\gluino$                                       &$g$    \\
                                                 & $-$               &$-$                    & $\wino$                                         &$W$    \\
                                                 & $-$               &$-$                    & $\bino$                                         &$B$    \\\hline
        \multirow{2}{*}{higgs \& higgsinos}      & $-$               &$(H^{+}_{u},H^{0}_{u})$& $(\widetilde{H}^{+}_{u},\widetilde{H}^{0}_{u})$ &$-$    \\
                                                 & $-$               &$(H^{0}_{d},H^{-}_{d})$& $(\widetilde{H}^{0}_{d},\widetilde{H}^{-}_{d})$ &$-$    \\
\hline\hline
\end{tabular}
\end{table}                                                                                                                                                                                             
%\begin{table}[ht!]
%\def\arraystretch{1.2}
%\setlength{\belowcaptionskip}{6pt}
%\small
%\centering
%\caption{An overview of the superpartners of the SM bosons and fermions \cite{Martin:1997ns}.}
%\label{tab:susy}
%\begin{tabular}{l c c c c c}
%        \hline \hline
%        SM \& SUSY particle              & chirality         &spin-0                 & spin-$\frac{1}{2}$                              &spin-1 &\small{$(SU(3)_{C},SU(2)_{L},U(1)_{Y})$}\\\hline
%        \multirow{2}{*}{leptons \& sleptons}     & $L$               &($\slepL,\sneuL$)      & ($\nu_{L}$,$\lepL$)                             &$-$    & $(1,2,-\frac{1}{2})$ \\
%                                                 & $R$               &$\slepR$               & $\lepR$                                         &$-$    & $(1,1,-1)$ \\\hline
%        \multirow{3}{*}{quarks \& squarks}       & $L$               &($\supL,\sdownL$)      & ($\upL$,$\downL$)                               &$-$    & $(3,2,\frac{1}{6})$ \\
%                                                 & \multirow{2}{*}{R}&$\supR$                & $\upR$                                          &$-$    & $(3,1,\frac{2}{3})$ \\
%                                                 &                   &$\sdownR$              & $\downR$                                        &$-$    & $(3,1,-\frac{1}{3})$ \\\hline
%        \multirow{3}{*}{gauge bosons \& gauginos}& $-$               &$-$                    & $\gluino$                                       &$g$    & $(8,1,0)$ \\
%                                                 & $-$               &$-$                    & $\wino$                                         &$W$    & $(1,3,0)$ \\
%                                                 & $-$               &$-$                    & $\bino$                                         &$B$    & $(1,1,0)$ \\\hline
%        \multirow{2}{*}{higgs \& higgsinos}      & $-$               &$(H^{+}_{u},H^{0}_{u})$& $(\widetilde{H}^{+}_{u},\widetilde{H}^{0}_{u})$ &$-$    & $(1,2,\frac{1}{2})$ \\
%                                                 & $-$               &$(H^{0}_{d},H^{-}_{d})$& $(\widetilde{H}^{0}_{d},\widetilde{H}^{-}_{d})$ &$-$    & $(1,2,-\frac{1}{2})$ \\
%\hline\hline
%\end{tabular}
%\end{table}                                                                                                                                                         
One striking feature from Table \ref{tab:susy} is the existence of more than one SM Higgs boson. 
SUSY $requires$ the existence of two Higgs doublets, $(H^{+}_{u},H^{0}_{u})$ and $(H^{+}_{d},H^{0}_{d})$, ordered in $SU(2)_{L}$ doublet complex scalar fields.\footnote{This might seem like a bold prediction, requiring four more Higgs bosons, but it turns out that the lightest Higgs boson could indeed be compatible with the 125\GeV one discovered. \cite{Ellis:1638463}}
The neutral scalar Higgs boson that we know and have discovered would be a linear combination of the $H^{0}_{u}$ and $H^{0}_{d}$.  
To each of these SM Higgs bosons is a superpartner, a $higgsino$, associated to it. 
\newpara
\noindent\justify
Another thing to notice in Table \ref{tab:susy} is the SM gauge bosons. 
Following the discussion in the previous chapter, the usual $\PW$, $\PZ$ and $\gamma$ bosons are the mass eigenstates of the gauge eigenstates $W_{1,2,3}$ and $B$. 
For this reason, the superpartners of the $W$ and $B$ are the ones listed, the so called $wino$ and $bino$. 
Similarly to how the $\PW$, $\PZ$ and $\gamma$ bosons are mixtures of $W_{1,2,3}$ and $B$, the winos and binos mix to form the mass eigenstates \firstcharg, \secondcharg\firstchi, \secondchi and so on. 
The ones with electric charge are called $charginos$, and the ones that are neutral, $neutralinos$.   
Only the mass eigenstates are observable via experiments. The mass eigenstates of the SUSY particles are listed in Table \ref{tab:eigenstates}.  
\begin{table}[ht!]
\def\arraystretch{1.2}
\setlength{\belowcaptionskip}{6pt}
\small
\centering
\caption{An overview of the gauge eigenstates and the mass eigenstates of the SUSY particles. \cite{Martin:1997ns}.}
\label{tab:eigenstates}
\begin{tabular}{l c c}
        \hline \hline
                                 & gauge eigenstates                                                 & mass eigenstates \\\hline
        \multirow{3}{*}{squarks} & $\supL$, $\supR$, $\sdownL$, $\sdownR$                            & $\supL$, $\supR$, $\sdownL$, $\sdownR$\\
                                 & $\scL$, $\scR$, $\ssL$, $\ssR$                                    & $\scL$, $\scR$, $\ssL$, $\ssR$\\
                                 & $\stL$, $\stR$, $\sbL$, $\sbR$                                    & $\stO$, $\stT$, $\sbO$, $\sbT$\\\hline
        \multirow{3}{*}{sleptons}& $\seL$, $\seR$, $\sneuE$                                          & $\seL$, $\seR$, $\sneuE$\\
                                 & $\smuL$, $\smuR$, $\sneuM$                                        & $\smuL$, $\smuR$, $\sneuM$\\
                                 & $\stauL$, $\stauR$, $\sneuT$                                      & $\stauO$, $\stauT$, $\sneuT$\\\hline
        gluino                   & $\gluino$                                                         & $\gluino$\\\hline
        gravitino                & $\gravitino$                                                      & $\gravitino$\\\hline
        neutralinos              & $\bino$, $\wino$, $\widetilde{H}^{0}_{u}$, $\widetilde{H}^{0}_{d}$& $\firstchi$, $\secondchi$, $\thirdchi$, $\fourthchi$\\\hline
        charginos                & $\wino^{\pm}$, $\widetilde{H}^{+}_{u}$, $\widetilde{H}^{-}_{d}$   & $\firstcharg$, $\secondcharg$\\
\hline\hline
\end{tabular}
\end{table}                                                                                                                                                                                             
For the heaviest of the fermions, the $t$ and $b$ quarks and the $\tau$ leptons, the superpartners are mixed to form mass eigenstates. 
Sfermion mixing for the first two families is assumed to be negligible.  
If all other features than the spin is the same for the SM and SUSY particles, the masses would also be the same. 
But as the superpartners have not been discovered at the same masses as their SM partners, there must be some mechanism included that gives the SUSY particles different masses. 
As we are already accustomed to the broken symmetry of the electroweak theory, it is clear that SUSY, if it exists, is a broken symmetry. 
\subsection*{$R$-parity}
\noindent\justify
Recalling the baryon and lepton numbers ($B$ and $L$) that were introduced in the previous chapter. 
In the SM, the baryon and lepton numbers need to be conserved in an interaction.\footnote{Consider a decay of a muon: $\mu^{-} (L_{e}=0, L_{\mu}=+1)\rightarrow e^{-} (L_{e}=+1, L_{\mu}=0) +\bar{\nu_{e}} (L_{e}=-1,L_{\mu}=0)+\nu_{\mu} (L_{e}=0,L_{\mu}=+1)$, the lepton numbers before and after are the same $L_{e}=0, L_{\mu}=+1$.}  
The MSSM can allow for interactions that violate either baryon or lepton number, although no such processes have been observed. 
Instead, a new quantity is formulated, denoted $R$ \cite{Farrar:1978xj}, defined using the baryon and lepton numbers, as well as the spin $s$:
\begin{equation}
R={3(B-L)+2s}.
\end{equation}
From this quantity, the $R$-parity, $P_{R}$, is defined as $P_{R}=(-1)^{R}$.
For SM particles, $P_{R}$ takes a value of +1, whereas for SUSY particles it takes a value of -1. 
The MSSM is defined such that $R$-parity is conserved in each interaction vertex, and the conservation leads to a couple of experimental implications. 
The first consequence is that SUSY particles are produced in pairs. 
The second consequence is that a SUSY particle must decay into another SUSY particle and a SM particle, when it is kinematically allowed. 
When it is not, i.e. when there are no lighter SUSY particles for a SUSY particle to decay into, this one particle is known as the $lightest$ $supersymmetric$ $particle$, or LSP. 
This feature of MSSM is what makes it a favorable theory in explaining the origin of dark matter. 
The LSP is electrically neutral, only interacts weakly with ordinary matter and is stable due to the $R$-parity conservation; hence it is an attractive candidate for dark matter.
The MSSM is a $R$-parity conserving model (RPC), but there exist models that violate $R$-parity (RPV), that are lacking the features described above.
\subsection*{The MSSM Lagrangian}
\noindent\justify
Similarly to what was done in the previous chapter, the theory can be written down in terms of a Lagrangian \cite{Pape:2006ar}. 
Neglecting for a second that MSSM must be a broken symmetry, the starting point for writing down the SUSY Lagrangian is:
\begin{equation} 
\mathcal{L}_{\mathrm{SUSY}}=\mathcal{L}_{\mathrm{int.}}+\mathcal{L}_{\mathrm{chiral.}}- V(\phi)
\end{equation}
The first term represents the kinetic terms and the gauge interactions, that can be written as:
\begin{equation} 
\mathcal{L}_{\mathrm{int.}}=-\mathcal{D}^{\mu}\phi^{*i}\mathcal{D}_{\mu}\phi_{i}-i\psi^{\dagger i}\bar{\sigma}^{\mu}\mathcal{D}_{\mu}\psi_{i}\\
                   -\frac{1}{4}F_{\mu\nu}^{4}F^{\mu\nu a}-i\lambda^{\dagger a}\bar{\sigma}^{\mu}\mathcal{D}_{\mu}\lambda^{a}\\
                   -\sqrt{2}g[(\phi^{*i}T^{a}\psi_{i})\lambda^{a}+\lambda^{\dagger a}(\psi^{\dagger i}T^{a}\phi_{i})].
\label{eq:susyL}
\end{equation}
The indices $i$ run over all chiralities and flavors, the $a$ is a label for the gauge bosons, and the $\mathcal{D}_{\mu}\phi_{i}$, $\mathcal{D}_{\mu}\psi_{i}$ and $\mathcal{D}_{\mu}\lambda^{a}$ are the covariant derivatives written as:
\begin{align} 
\mathcal{D}_{\mu}\phi_{i}=&\partial_{\mu}\phi_{i}+igA_{\mu}^{a}(T^{a}\phi)_{i},\\
\mathcal{D}_{\mu}\psi_{i}=&\partial_{\mu}\psi_{i}+igA_{\mu}^{a}(T^{a}\psi)_{i},\\
\mathcal{D}_{\mu}\lambda^{a}=&\partial_{\mu}\lambda^{a}-g f^{ABC}A_{\mu}^{B}A_{\nu}^{C}.
\end{align}
The $T^{a}$ are the generators of the gauge group. 
The second line in Equation \ref{eq:susyL} contains the same field strength tensor as introduced in Equation \ref{eq:fst}, that yields the kinetic terms of the gauge fields.
The second term in the $\mathcal{L}_{\mathrm{SUSY}}$ is the MSSM Lagrangian for chiral fields $\mathcal{L}_{\mathrm{chiral.}}$ that is written as 
\begin{align} 
\mathcal{L}_{\mathrm{chiral.}}=-\frac{1}{2}\frac{\partial^{2}W}{\partial \phi_{i}\partial \phi_{j}}\psi_{i}\psi_{j}+\mathrm{hermitian\,\,conjugate}.
\end{align}
The $W$ in the chiral term is the so called $superpotential$ that in MSSM takes the form:
\begin{align} 
W=\epsilon_{ij}(-\bar{L}_{L}^{i}h_{L}\bar{E}_{L}^{C}H_{d}^{j}-\bar{Q}_{L}^{i}h_{D}\bar{D}_{L}^{C}H_{d}^{j}+\bar{Q}_{L}^{i}h_{U}\bar{U}_{L}^{C}H_{d}^{j}+\mu H_{u}^{i}H_{d}^{j})
\end{align}
with $i$, $j$ being isospin indices, $\epsilon_{ij}=-\epsilon_{ji}$, $h_{L}$, $h_{D}$, $h_{U}$ are Yukawa couplings and the $\mu$ is a Higgs mixing parameter.
$R$-parity that was introduced in the previous section, is conserved in the superpotential of MSSM. 
The final term in the $\mathcal{L}_{\mathrm{SUSY}}$ is the MSSM scalar potential, $V(\phi)$. 
It contains two contributions, the chiral contribution and the gauge contribution, that can be written in the following way:
\begin{align}
F_{i}=&\frac{\partial W}{\partial \phi_{i}},\\
D^{a}=&-g\phi^{*i}T_{ij}^{a}\phi_{j},
\end{align}
respectively. The final expression for the potential is:
\begin{align}
V(\phi)=F_{i}^{*}F_{i}+\frac{1}{2}D^{a}D^{a}
\end{align}
The $\mathcal{L}_{\mathrm{SUSY}}$ will be the starting point for the next section, where the breaking of the SUSY is presented. 
\section{SUSY breaking mechanisms}\label{sec:breaking}
\noindent\justify
As the superpartners have not been discovered at the same masses as their SM partners, SUSY is a broken symmetry. 
In order to accommodate for this required breaking of the symmetry while still maintaining the desirable feature of SUSY that cancels the divergences in Equation \ref{eq:loop}, a so called $soft$ SUSY breaking scenario is considered. 
This is done by adding a new term, a soft SUSY breaking term, to the $\mathcal{L}_{SUSY}$ \cite{soft,Chung:2003fi}:
\begin{equation}
\mathcal{L}=\mathcal{L}_{\mathrm{SUSY}}+\mathcal{L}_{\mathrm{soft}}
\label{eq:softterm}
\end{equation}
The $\mathcal{L}_{\mathrm{SUSY}}$ is the same as introduced previously, and preserves the supersymmetric interactions. 
%\begin{equation}
%\mathcal{L}_{\mathrm{soft}}=-\sum_{\tilde{q},\tilde{l},H_{u,d}}m_{0,i}^{2}|\phi_{i}|^{2}+(-\frac{1}{2}m_{1/2,a}\lambda_{a}\lambda_{a}-A_{0,i}W_{3,i}-B_{0}\mu H_{u}H_{d})+\mathrm{h.c}    
%\label{eq:softterm}
%\end{equation}
%Here the $m_{0,i}$ are the mass terms of the scalars and $m_{1/2,a}$ of the gauginos, $A_{0,i}$ and $B_{0}$ are parameters with (positive) dimension of mass and $W_{3,i}$ represents trilinear terms of the superpotential \cite{Pape:2006ar}.  
The soft terms include the introduction of mass parameters that lift the mass degeneracy of the SM particles and their superpartners. 
As these parameters contribute to the radiative corrections discussed previously, they need to be less than a few \TeV in order to still solve the hierarchy problem. 
Following the description in \cite{Martin:1997ns}, it is evident that SUSY breaking should occur in a “hidden sector” of particles that have very little direct couplings to the “visible sector” chiral supermultiplets of the MSSM. 
The question is what these mediating interactions should be, that is the connection between the hidden sector and the visible sector. 
There are countless proposals to the design of the mediating interactions in the soft term on the market, of which the two most popular will be introduced. 
\subsection*{SUSY breaking in SUGRA}
\noindent\justify
The first proposal for what the mediating interactions are dates back to the beginning of the 1980s \cite{Chamseddine:1982jx,Barbieri:1982eh,Ibanez:1982ee,Hall:1983iz,Ellis:1982wr}.
It proposes that the mediating force is gravity, hence the name SUGRA, short for SUperGRAvity theory, generalized to many different theories in various numbers of dimensions and involving $N$ supersymmetries. 
mSUGRA means minimal supergravity theory, and is a special case of SUGRA with $N=1$. 
%There is a direct coupling of the goldstino supermultiplets to the MSSM states with gravitational strength \cite{Pape:2006ar}.
If SUSY is broken in the hidden sector by a vacuum expectation value $\braket{F}$ then soft terms in the Lagrangian becomes
\begin{equation}
m_{\mathrm{soft}}\approx\frac{\braket{F}}{M_{P}}
\end{equation}
where $M_{P}$ is the reduced Planck mass. 
For $m_{\mathrm{soft}}$ of order a few hundred \GeV, the expected􏰛scale associated with the origin of SUSY breaking in the hidden sector should be around $\sqrt{\braket{F}}\approx10^{10}-10^{11}\GeV$.
In mSUGRA the gravitino is supposed to be heavy (same order as $m_{\mathrm{soft}}$, interacting very weakly, and the LSP is most likely a neutralino). 
\subsection*{SUSY breaking in GMSB}
\noindent\justify
Gauge mediated supersymmetry breaking \cite{Dine:1981gu,Nappi:1982hm,Dine:248065}, or GMSB for short, is a model that assumes the mediating interactions between the hidden sector and the visible sector to be ordinary electroweak and QCD gauge interactions with extra chiral supermultiplets. 
As opposed to SUGRA where the communication between the hidden sector and the visible sector happen through a mediator, in GMSB the hidden sector couple to a messenger sector instead of directly with the visible sector.
In this scenario the soft term arise due to loop diagrams that include a messenger particle. 
The messengers are chiral supermultiplets that couple to the $\braket{F}$ and have the ordinary $SU(3)_{C}\otimes SU(2)_{L}\otimes U(1)_{Y}$ interactions. 
The soft term is now
\begin{equation}
m_{\mathrm{soft}}\approx\frac{\alpha_{a}}{4\pi}\frac{\braket{F}}{M_{\mathrm{messenger}}}
\label{eq:gmsb}
\end{equation}
where the $\frac{\alpha_{a}}{4\pi}$ stems from the ordinary gauge interactions and the $M_{\mathrm{messenger}}$ is the mass of the messenger. 
The immediate effect of Equation \ref{eq:gmsb} is that in order to yield a soft mass of the order of the weak scale, the scale of the SUSY breaking should be as low as $\sqrt{\braket{F}}\approx10^{4}-10^{5}\GeV$, which is much lower than that obtained in mSUGRA.
As an effect of this low energy scale, the gravitino is assumed to be very light ($\approx1\eV$) and is thus always the LSP. 
\newpara
\noindent\justify
What might not obvious from the discussions throughout this chapter is that the unconstrained MSSM contains a huge number of free parameters (105 in addition to the 19 free parameters of the SM!) \cite{Dimopoulos:1995ju}.
In order to be able to use the model in a reasonable way, one needs a way to reduce the number of parameters \cite{Djouadi:1998di}. 
At hand are a couple of phenomenological constraints that can be used to bring down the number of parameters, that will be listed in the following. 
Recalling the $\mathcal{CP}$ violation that was introduced in Section \ref{sec:yukawa}. 
There is some evidence for $\mathcal{CP}$ violation, but the parameters of the MSSM can be reduced by assuming that there are no new sources of $\mathcal{CP}$ violation.  
Further, an assumption is made that there are no so called flavor changing neutral currents, FCNC. 
Finally, the assumption that the soft SUSY breaking scalar masses are the same for the first and second generations \cite{Djouadi:1998di}.
By imposing these phenomenological constraints, the number of parameters are reduced to only 19. 
This model is known as phenomenological MSSM, or pMSSM for short.  
