\chapter{Search for Charginos, Neutralinos and Sleptons} \label{strategySlepton}
After having introduced the CMS detector, its abilities of reconstructing a wide variety of SM particles and the data processing, it is time to present what all this can be used for. 
The theoretical framework presented in~\ref{theory} lays the foundation for the search for electroweak production of SUSY particles that this thesis is presenting. 
This thesis is relying on two papers, one in which a general search for strongly and electroweakly produced particles is performed, in final states containing two opposite sign same flavor leptons, large \ptmiss and more than two jets. 
As all work performed in big collaborations such as the CMS experiment, no paper can be published independently. 
Instead, the work of many people at different institutes is required, and all authors of the CMS experiment contribute to the collection of the data, validation fo reconstruction algorithms, derivation of corrections, developing of analysis methods, to name a few. 
With this in mind, it is appropriate to mention that the first paper presented in this thesis relies on a collaboration of researchers at ETH Z\"{u}rich, RWTH Aachen, University of California at San Diego and University of Oviedo. 
I will highlight my contributions to this paper, and cite the work of my collaborators when necessary. 
Since my main contribution to the first paper was in the search for chargino-neutralino (\firstcharg-\secondchi) and neutralino-neutralino (\firstchi-\firstchi) production, I will dedicate a large fraction of this thesis to a description of this search. 
The second paper that this thesis is based on, is a search for directly produced selectrons ($\seL$, $\seR$) and smuons ($\smuL$, $\smuR$). 
This search has many similarities to the search for \firstcharg-\secondchi and \firstchi-\firstchi production, but has some main differences that needs other SM background prediction techniques. 
This paper can be seen as a spin-off to the general analysis searching for SUSY in opposite sign dilepton pairs performed by ETH Z\"{u}rich, while still acknowledging the commonalities with the collaborative search. 
This chapter will contain a brief overview of the different SUSY particle production modes. 
Subsequent to this chapter is one that contains the description of the triggers, objects and background prediction techniques that are common to the two searches. 
After the common elements of the searches have been presented, the two searches are presented separately. 
Following the order dictated by the date of publication, the analysis strategy and results for the search for \firstcharg-\secondchi and \firstchi-\firstchi production are first presented. 
The full result is published in \ref{edge2016}. 
The search for selectrons and smuons is published in \ref{sleptons2016}.
\section{Charginos, neutralinos and sleptons}
The SUSY models considered in the first paper assume electroweak production (EW) and Gauge mediated SUSY breaking (GMSB), and are presented in Fig.~\ref{sig:feynmanEWK}. 
The \firstcharg-\secondchi production is depicted in the upper part of Fig.~\ref{sig:feynmanEWK}. 
In this model, the \firstcharg is set to decay to a \PW boson and a \firstchi, which is the LSP, while the next-to-lightest neutralino, \secondchi, decays to a \PZ boson and \firstchi.
The production cross sections for this model are computed in a limit of mass-degenerate wino \firstcharg\ and \secondchi, and light bino \firstchi.  
All the other SUSY particles are assumed to be heavy and decoupled.
The remaining two models considered in the first paper assume the production of \firstchi-\firstchi pairs in GMSB.
For bino- or wino-like neutralinos, the neutralino pair production cross section is very small, and thus a specific GMSB model is considered, with mass-degenerate higgsinos \firstcharg, \secondchi, and \firstchi as the next-to-lightest SUSY particles and a massless gravitino as the LSP~\cite{Matchev:1999ft,Meade:2009qv,Ruderman}.
In the production of any two of these, \firstcharg\ or \secondchi\ decays immediately to \firstchi\ and low-momentum particles that do not impact the analysis, effectively yielding pair production of $\firstchi\firstchi$.
Intermediate production of either \firstcharg\ or \secondchi\ is therefore not explicitly shown in the lower two diagrams of Fig.~\ref{sig:feynmanEWK} representing these models.
In the first model (lower left of Fig.~\ref{sig:feynmanEWK}), the only allowed decay of the lightest neutralino is to a \PZ boson and a massless gravitino.
In the other model (lower two diagrams of Fig.~\ref{sig:feynmanEWK}), the lightest neutralino is allowed to decay to a gravitino and either a \PZ boson or an SM-like Higgs boson, with a 50\% branching fraction to each decay channel.
The cross sections for higgsino pair production are computed in a limit of mass-degenerate higgsino states \secondchi, \firstcharg, and \firstchi.
Again, all the other SUSY particles are assumed to be heavy and decoupled.
Following the convention of real mixing matrices and signed neutralino masses~\cite{Skands:2003cj}, the sign of the mass of \firstchi (\secondchi) are set to $+1$ ($-1$).
The lightest two neutralino states are defined as symmetric (anti-symmetric) combinations of higgsino states by setting the product of the elements $N_{i3}$ and $N_{i4}$ of the neutralino mixing matrix $N$ to $+0.5$ ($-0.5$) for $i = 1$ ($2$).
The elements $U_{12}$ and $V_{12}$ of the chargino mixing matrices $U$ and $V$ are set to 1.
Common to these models is that they produce an on-shell \PZ boson giving the OCSF leptons, in association with a SM boson, and two LSPs resuting in large \ptmiss. 
The SM boson is either a \PW, \PZ or a H, and their decay mode to jets (or b-tagged jets in the case of the H) is targeted in the search.   
\begin{figure}[!h]
\centering
\includegraphics[width=0.45\textwidth]{images/ewk/Figure_001-b.pdf}\\
\includegraphics[width=0.45\textwidth]{images/ewk/Figure_001-c.pdf}
\includegraphics[width=0.45\textwidth]{images/ewk/Figure_001-d.pdf}
\caption{(Upper) Diagram corresponding to the chargino-neutralino production with the \firstcharg and \secondchi decaying into vector bosons and the LSP. 
(Lower) Diagrams corresponding to the neutralino-neutralino model of where the neutralinos are allowed to decay to a gravitino and a \PZ boson (left) and where the neutralinos are allowed to decay to a gravitino and a \PZ boson or a Higgs boson, with a 50\% branching fraction to each decay channel (right).}
\label{sig:feynmanEWK}
\end{figure}                                                                                                                                 
Furhter, SUSY models predict charged sleptons ($\seL$, $\smuL$, $\stauL$, $\seR$, $\smuR$, $\stauR$), the superpartners of the charged left-handed and right-handed SM leptons. 
At sufficiently heavy slepton masses, the sleptons undergo a two-body decay into one of the heavier neutralinos or a chargino, while direct decays to a neutralino LSP are favored for light slepton masses.
The second paper that this thesis is based on is on a search for directly produced selectrons and smuons, under the assumption of direct decays $\slep\to \ell\lsp$ with 100\% branching ratio, as sketched in Fig.~\ref{fig:feynmanSlepton}.
\begin{figure}[!h]
\centering
\includegraphics[width=0.55\textwidth]{images/slepton/Figure_001.pdf}
\caption{Diagram corresponding to the slepton model with two selectrons (smuons) directly produced and decay into electrons (muons) and a LSP} 
\label{sig:feynmanSlepton}
\end{figure}                                                                                                                                 
The final states resulting from directly produced selectrons (smuons) are a pair of electrons (muons) and large \ptmiss from the LSPs. 
The main differences in the final state compared to the \firstcharg-\secondchi and \firstchi-\firstchi production, are that absence of jets and the fact that the lepton pair is not compatible to the \PZ boson mass. 
These two distinctions make for a different search strategy, where no jets are required, and the contribution from Drell--Yan can be greatly suppressed through a veto. 
But before focusing on the differences, there are many features that are very similar between these two searches, like the datasets, triggers, objects and some SM background prediction techniques. 
In the subsequent chapter, the common quantities will be presented.  
