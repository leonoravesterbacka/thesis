\chapter*{Zusammenfassung}
\noindent
\justify
In dieser Dissertation werden physikalische Prozesse jenseits des Standardmodells der \\Teilchenphysik (SM) gesucht. 
Die Analysen werden mit $35.9\,\mathrm{fb^{-1}}$ Proton Proton Kollisionsdaten durchgef\"{u}hrt, die 2016 mit dem CMS Detektor bei einer Schwerpunktsenergie von $13\TeV$ im LHC am CERN gesammelt wurden. 
Die neuen Physikph\"{a}nomene zielen auf starke und elektroschwache Produktion von supersymmetrischen (SUSY) Teilchen im Rahmen des minimal supersymmetrischen Standardmodells ab. 
Leistungsf\"{a}hige Suchstrategien nach SUSY Teilchen verwenden Endzustandsszenarien mit zwei Elektronen oder zwei Myonen entgegengesetzter elektrischer Ladung. 
\newpara
\noindent\justify
Dieses Manuskript enth\"{a}lt eine Zusammenfassung der theoretischen Modelle von SM und SUSY sowie eine umfassende Beschreibung des CMS Experiments am LHC Beschleunigerkomplex. 
Die Arbeit basiert auf zwei Analysen: Die erste zielt auf die starke und elektroschwache Produktion von Superpartnern ab, die zu einem leptonischen Endzustand zusammen mit mehreren Jets und gro{\ss}em fehlendem transversalen Impuls f\"{u}hren. 
Die zweite Analyse untersucht die direkte Produktion der Superpartner von SM-Elektronen und -Myonen, wobei ein rein leptonischer Endzustand mit gro{\ss}em fehlenden transversalen Impuls verwendet wird.
Beide Suchen haben die Verwendung des fehlenden transversalen Impulses gemein, der eine entscheidende Variable f\"{u}r die Suche nach $R$-Parity erhaltender SUSY ist. 
Diese Dissertation enth\"{a}lt ein Kapitel \"{u}ber die Leistung des Rekonstruktionsalgorithmus des fehlenden transversalen Impulses sowie einen \"{U}berblick \"{u}ber die Suchstrategien und die Absch\"{a}tzung der SUSY Produktionsmodus spezifischen SM Hintergrundprozesse.
\newpara
\noindent\justify
Da bei keiner der Suchen ein \"{U}berschuss an Kollisionsdaten in Bezug auf die vorhergesagten SM-Hintergrundprozesse beobachtet wird, wird eine statistische Interpretation der Ergebnisse durchgef\"{u}hrt, um Obergrenzen f\"{u}r die Produktionsquerschnitte der SUSY Teilchen festzulegen. 
Diese neuen Grenzwerte erweitern die w\"{a}hrend des LHC Run 1 unter Verwendung von 8 TeV Kollisionsdaten gemesssenen Grenzwerte erheblich. 
