\part{THEORY AND EXPERIMENTAL APPARATUS}
\noindent\justify
\chapter{The Standard Model of Particle Physics} \label{sec:theory}
\noindent\justify
The theoretical motivation behind this thesis is summarized in the following, very well used, sentence: 
The Standard Model of particle physics (SM) is an extremeley successful theory that is able to correctly describe almost all known particles, forces and phenomena that make up the Universe. 
The success of the theory has been the countless discoveries and measurements of the predicted elementary particles throughout the last half of the 20th century and beginning of the 21st century.  
The most recent triumph of the SM was the discovery of the predicted Higgs boson by the ATLAS \cite{Aad:2012tfa} and CMS \cite{Chatrchyan:2012xdj} collaborations in 2012.  
The caveat lies in the word $almost$. 
There are phenomena that the SM is unable to explain, like the origin of dark matter and the non-zero neutrino masses. 
This chapter provides an overview of all the SM particles and an introduction to the theoretical framework. 
The convention is to formulate particle physics by giving the SM Lagrangian, and using quantum field theory all measurable quantities such as mass and cross sections can be calculated. 
Before the SM Lagrangian can be formulated, the concept of symmetries and their relationship to groups has to be introduced.
After the SM Lagrangian has been formulated, the chapter concludes with the motivations for an extension of the theory in order to account for the unexplained phenomena.
\newpage
\section{Particles and interactions}  
\noindent\justify
All known matter is built by fundamental particles, and more precisely of so called $fermions$. 
The fundamental interactions that act between the fermions are mediated by $bosons$. 
Another way to categorize the paricles is by their spin. Fermions have half integer spin, whereas bosons have integer spin. 
\subsection*{Particle content of the SM}  
\noindent\justify
The most familiar fundamental particles are the electrons, and the up and the down quarks. 
These are the building blocks that make up all everyday matter. 
The quarks are combined into protons and neutrons that in turn can combine into nuclei, that create atoms when combined with electrons. 
Atoms and molecules account for all known matter around us. 
The electron and the up and down quark are collectively referred to as fermions, as they have the common feature of having half integer spin. 
The electron is a part of a larger $family$ of particles called $leptons$, and the up and down quarks are members of the $quark$ family.
Both families of fermions come in three different generations, which are ordered by their mass. 
Further, the organization of the SM particles associates an electron, $e^{-}$, to its $electron$ $neutrino$, $\nu_{e}$, and this pair is the first of the generations of the lepton family, where the heavier muon lepton, $\mu^{-}$, and its muon neutrino, $\nu_{\mu}$, is the second generation, and the heaviest tau lepton $\tau^{-}$ and its tau neutrino $\nu_{\tau}$ is the third generation.  
The up and down quarks, $u$ and $d$, are the first generation of quarks, followed by the heavier charm and strange quarks, $c$ and $s$ in the second generation, and the heaviest top and bottom quarks,$t$ and $b$. 
The difference between quarks and leptons is that quarks carry strong charge $Y_{s}$ whereas leptons do not. 
To each of the fermions introduced is a corresponding antiparticle, doubling the population of the particle zoo. 
The antiparticle carries opposite electric charge $Q$ to the particle, and is denoted either by the sign of the charge or by a bar. 
I.e. the antiparticle of an electron $e^{-}$ is $e^{+}$ (called positron), and the antiparticle of an up-quark $u$ is $\bar{u}$ (called anti-up). 
Further, all the fermions come in two chirality states, denoted as $right$ or $left$-handed, with a subscript of $R$ or $L$. 
In the same way that the electrical charge is the defining difference between a particle and its antiparticle, so is the so called $weak$ charge different between $right$ and $left$-handed particles. 
The third component of the weak isospin is denoted $T_{3}$, and this quantity is used to relate the so-called weak hypercharge $Y_{W}$ to the charge $Q$ through the relationship $Y_{W}=2(Q-T_{3})$. 
Neutrinos are a special case of this ordering. 
As they do not carry electric charge, the antiparticle of the neutrino is ill-defined. 
Further, the right handed chirality state does not carry weak charge. 
Table \ref{tab:fermions} facilitates the overview of the paricles introduced so far.
\begin{table}[ht!]
\def\arraystretch{1.2}
\setlength{\belowcaptionskip}{6pt}
\small
\centering
\caption{}
\label{tab:fermions}
\begin{tabular}{l c c c c c c c}
        \hline \hline
        & \multicolumn{3}{c}{Generation} &\multicolumn{4}{c}{Charge} \\
        & First & Second & Third & $Q$ & $T_{3}$ & $Y_{W}$ & $Y_{S}$ \\\hline
\multirow{3}{*}{Charged leptons}& electron ($0.5\MeV$) & muon ($106\MeV$) & tau ($1.78\GeV$)&  & & & \\
                        & $e_{L}$ & $\mu_{L}$ & $\tau_{L}$ & -1 & $\frac{1}{2}$ & -1 & no\\
                        & $e_{R}$ & $\mu_{R}$ & $\tau_{R}$ & -1 & 0             & -2 & no\\\hline
\multirow{2}{*}{Neutral leptons}& electron $\nu$ ($<\eV$)& muon $\nu$ ($<\eV$)& tau $\nu$ ($<\eV$)&  & & & \\
                        & $\nu_{e}$ & $\nu_{\mu}$ & $\nu_{\tau}$ & 0 & $-\frac{1}{2}$ & -1 & no\\\hline
\multirow{3}{*}{Up-type quarks} & up ($2.3\MeV$)& charm ($1.29\GeV$)& top ($173\GeV$)&  & & & \\
                        & $u_{L}$ & $c_{L}$ & $t_{L}$ & $+\frac{2}{3}$ & $\frac{1}{2}$ & $\frac{1}{3}$ & yes\\
                        & $u_{R}$ & $c_{R}$ & $t_{R}$ & $+\frac{2}{3}$ & 0 & $\frac{4}{3}$ & yes\\\hline
\multirow{3}{*}{Down-type quarks} & down ($4.8\MeV$)& strange ($95\MeV$)& bottom ($4.8\GeV$)&  & & & \\
                        & $d_{L}$ & $s_{L}$ & $b_{L}$ & $-\frac{1}{3}$ & $-\frac{1}{2}$ & $\frac{1}{3}$ & yes\\
                        & $d_{R}$ & $s_{R}$ & $b_{R}$ & $-\frac{1}{3}$ & 0 & $-\frac{2}{3}$ & yes\\\hline
\hline
\end{tabular}
\end{table}                                                                                                              
\subsection*{Fundamental interactions} 
\noindent\justify
The other set of particles in the SM, the bosons, have integer spin. 
Between the fermions introduced in Table \ref{tab:fermions} act fundamental forces. 
The forces are carried, or mediated, through gauge bosons which are responsible for the forming of more complex entities, like atoms or people. 
The four spin 1 force carriers are the gluons, photon, the $\PW^{+}/\PW^{-}$ bosons and the \PZ boson. 
The gluons mediate the strong force between particles with color charge, i.e. the quarks.
The photon mediate the electromagnetic force between particles with electric charge, i.e. the leptons, the quarks and the charged gauge bosons. 
The $\PW^{+}/\PW^{-}$ bosons and \PZ mediate the weak force between particles with isospin or weak hypercharge $Y_{W}$, such as the charged and neutral leptons, and the quarks.
The gauge bosons and their properties are summarized in Table \ref{tab:bosons}. 
\begin{table}[ht!]
\def\arraystretch{1.2}
\setlength{\belowcaptionskip}{6pt}
\small
\centering
\caption{}
\label{tab:bosons}
\begin{tabular}{l c c c c }
        \hline \hline
        Name & Symbol & Interaction & Electromagnetic charge & Mass \\\hline
        \PZ-boson   & \PZ & Weak & 0 & 91.2\GeV \\
        \PW-boson   & $\PW^{+}/\PW^{-}$ & Weak & $+/-$ & 80.4\GeV \\
        Photon      & $\gamma$          & Electromagnetic & 0 & 0\\
        Gluon       & $g$               & Strong          & 0 & 0\\
        \hline \hline
\end{tabular}
\end{table}                                                                    
Common to the gauge bosons listed in Table \ref{tab:bosons} is that they all are spin-1 bosons. 
In addition to the spin-1 bosons, the SM predicts a scalar (spin-0) boson, the Higgs boson. 
This boson does not carry any force like the spin-1 bosons, but is responsible for giving masses to all the other bosons and the fermions, known as the $Higgs$ $mechanism$. 
This section serves as an introduction to the SM particles and their properties. 
What has seemed like an ad hoc organization of the masses and the couplings of the particles, in fact rely on a sound theoretical framework. 
In the next chapter, the SM Lagrangian is introduced, that will provide more rigor to the chapter.  
\section{Standard Model Lagrangian}
\noindent\justify
The mathematical framework used to formulate the SM is quantum field theory (QFT), that treats the previously introduced particles as fields, represented by wave functions. 
QFT enables calculation of measurable probabilities, such as the probability for two particles to collide and produce another set of particles, and the probability for a particle to decay into other particles, and the mass of the particles. 
These probabilities make the foundation that is used in experimental high energy physics, by enabling for calculation of for example cross sections and branching fractions. 
The fundamental forces, that are the starting points to calculate these probabilities, are written in terms of Lagrangian densities. 
\subsection*{Symmetries and groups}\label{sec:symmetries}
\noindent\justify
Before the SM Lagrangian density can be formulated, the concept of symmetries is crucial to introduce. 
In classical mechanics, the Lagrangian $\mathcal{L}$\footnote{The term Lagrangian refers to the Lagrangian density, $\int d^{4}x\mathcal{L}(x,t)$, but as it is obvious from the context what is meant, I will continue to refer to the Lagrangian density as the Lagrangian. } is used to describe the energy of a system. 
For non-relativistic particles, it can simply be defined as:
\begin{equation}
\mathcal{L}=T-V
\end{equation}
where $T$ is the kinetic energy, and $V$ is the potential energy. 
A transformation of a variable appearing in a Lagrangian $\mathcal{L}$, is called a symmetry of $\mathcal{L}$ if it leaves the $\mathcal{L}$ unchanged. 
Examples are rotational, translational, reflectional and Lorentz transformations. 
The classical field theory of electricity and magnetism is left unchanged after any or all of those transformations.\footnote{In realitiy, the statement is not harder to understand than the statement that a physics experiment should give the same outcome, regardless if we rotated the setup, if we are moving at some vconstant velocity or perform the experiment at another location.} 
A symmetry is called $global$ if there is no dependence on some spacetime coordinate $x$, $local$ if there is, making it a stronger requirement. 
One of the most important contributions to theoretical particle physics\footnote{This is the authors opinion.}, proven by Emmy Noether \cite{Noether1918}, dictates that there is a conserved physical quantity associated to each local transformation that leaves the Lagrangian invariant.   
To put this important contribution in context, a Lagrangian symmetric under rotational transformation dictates that the angular momentum of the system is conserved.
Similarly, the conservation of linear momentum is related to translational invariance, and energy conservation to time translational invariance.  
Symmetries are described through the mathematical structure of $groups$. 
A group is defined as a set of elements and a composition rule. 
The criteria are summarized as: the combination of two elements, $E$, should give another element, there exists an identitiy element $I$ such that $E\cdot I=I\cdot E=E$, there exists an inverse $E^{-1}$ such that $E\cdot E^{-1}=E^{-1}\cdot E=I$ and the composition rule (denoted by $\cdot$) is associative \cite{Kane:2244793}.   
Groups enter in to the formulation of particle physics as they describe the property of carrying out a transformation on a physical system that leaves the system unchaged. 
The most simple example of a group is the one dimensional unitary group, or $U(1)$ group. 
The set of all complex phases of a wave function $U(\theta)=e^{i\theta}$, with $\theta$ being a real parameter, form a group.
It is straight forward to verify that the criteria listed above hold. 
Consider that some theory be invariant under 
\begin{equation}
\Psi(\vec{x},t)\rightarrow \Psi'(\vec{x},t)=e^{-i\alpha}\Psi(\vec{x},t)
\end{equation}                                                         
with $\alpha$ being a constant (not depending on $\vec{x}$ or $t$)\footnote{This might seem like a too rigorous explanation for a thesis in experimental high energy physics, but the following example helps understand the more advanced groups that make up the SM theory.}. 
This is a global gauge transformation as $\Psi(\vec{x},t)$ transforms the same everywhere.  
If one instead considers that the $\alpha$ depends on some $\vec{x}$, $\alpha(\vec{x},t)$, the theory needs to be invariant under 
\begin{equation}
\Psi(\vec{x},t)\rightarrow \Psi'(\vec{x},t)=e^{-i\alpha(\vec{x},t)}\Psi(\vec{x},t). 
\label{eq:local}
\end{equation}                                                         
A Lagrangian is normally defined using some derivatives, for example as in the case of the momentum operator $\vec{p}/2m=-\nabla/2m$. 
As is obvious from Equation \ref{eq:local}, any derivative in the Lagrangian with a dependence on $\vec{x}$ will result in an additional term involving $\alpha(\vec{x},t)$ from letting the derivative act on $e^{-i\alpha(\vec{x},t)}$, thus disabling the invariance.
In order to recover the invariance, a trick is performed to sort of counter act this additional term. 
The idea is to introduce a field, $A_{\mu}$ that transforms as $A_{\mu}\rightarrow A_{\mu}'=A_{\mu}-\frac{\partial_{\mu}\alpha(\vec{x},t)}{e}$, in the $covariant$ $derivative$
\begin{equation}
\mathcal{D}_{\mu}=\partial_{\mu}-ieA_{\mu}.
\end{equation}                                                         
The covariant derivatives replaces the derivatives in the Lagrangian, and it can easily be shown that the the introduction of the field $A_{\mu}$ recovers the invariance. 
The result can be interpreted in the following way. 
Local phase invariance of a theory $requires$ the existence of a field $A_{\mu}$. 
And more importantly, one can associate this $A_{\mu}$ field, which is a four-vector, to a vector, i.e. a spin-1 particle. 
This example summarizes the principle at hand, by requiring local phase invariance of a group, the existence of an interaction field is required. 
This principle will be used to identify the interaction fields through the more complex group structure. 
\newline
The local symmetry group of the SM is $U(1)_{Y}\otimes SU(2)_{L}\otimes SU(3)_{C}$ and in the next section the electroweak forces, described by $U(1)_{Y}\otimes SU(2)_{L}$, will be introduced, followed by the strong force, described by the $SU(3)_{C}$ group. 
In the end, all the gauge bosons introduced in Table \ref{tab:bosons} will have been identified in this local symmetry group. 
\subsection*{The electroweak theory} 
\noindent\justify
In the same way that the invariance of the theory under the $U(1)$ transformation required the existence of a field $A_{\mu}$ and an associated spin-1 gauge boson, an internal invariance under a set of transformations that form a so called $SU(2)$ group will require the existence of more spin-1 gauge bosons. 
The $SU(2)$ group can be called the electroweak $SU(2)$ invariance, and will be shown to have the associated gauge bosons $W^{\mu}_{i}$ with $i=1$, $2$ or $3$.
In order to identify the gauge bosons $W^{\mu}_{i}$, we have to write down the $SU(2)_{L}\otimes U(1)_{Y}$ Lagrangian. 
The starting point when writing down the Lagrangian is the Dirac Lagrangian, 
\begin{equation}
\mathcal{L}=\bar{\psi}(i\gamma^{\mu}\partial_{\mu}-m)\psi
\end{equation}                                               
where the $\psi$ is the fermion field, the $\gamma^{\mu}$ are the $\gamma$-matrices, the $\bar{\psi}=\gamma^{0}\psi$ is the conjugate fermion field and the $m$ is the fermion mass. 
Particles with spin-$\frac{1}{2}$, fermions, are put in doublets\footnote{These doublets could have also been denoted as a lepton doublet $L=\begin{pmatrix} e^{-}\\ \nu_{e} \end{pmatrix}$, or quark doublet $Q=\begin{pmatrix} u\\ d \end{pmatrix}$}
\begin{equation}
\Psi=\begin{pmatrix} \psi_{1}\\ \psi_{2} \end{pmatrix}.
\end{equation}
Similarly to what was done in Section \ref{sec:symmetries}, the local invariance of the theory under the $SU(2)$ group is achieved by the addition of the term $i\bar{\Psi}\gamma_{\mu}\frac{\sigma_{i}}{2}W^{\mu}_{i}\Psi$. 
This term describes the interactions between three massive spin-1 fields $W_{i}^{\mu}$ and two spin-$\frac{1}{2}$ fields, where the $W_{i}^{\mu}$ transforms as $W_{i}^{\mu}\rightarrow W'^{\mu}_{i}=W_{i}^{\mu}+\partial_{\mu}a_{i}(x)$. 
Putting aside the mass term for a second, the Lagrangian can now be written as
\begin{equation}
\mathcal{L}_{SU(2)}=i\bar{\Psi}\gamma_{\mu}\partial^{\mu}\Psi+ \bar{\Psi}\gamma_{\mu}\frac{\sigma_{i}}{2}W_{i}^{\mu}\Psi-\frac{1}{4}(W_{\mu\nu})_{j}(W^{\mu\nu})_{j} 
\label{eq:dirac}
\end{equation}                                               
where $(W_{\mu\nu})_{j}=\partial_{\mu}(W_{\nu})_{j}-\partial_{\nu}(W_{\mu})_{j}$. 
Equation \ref{eq:dirac} is thus $SU(2)$ invariant.  
This locally $SU(2)$ invariant Lagrangian can be unified with a locally $U(1)$ invariant Lagrangian, namely
\begin{equation}
\mathcal{L}_{U(1)}=-m\bar{\psi}\psi+\bar{\psi}\gamma_{\mu}(i\partial^{\mu}+g\frac{Y}{2}B^{\mu})\psi-\frac{1}{4}B_{\mu\nu}B^{\mu\nu}
\end{equation}                                               
The $Y$ is the same as as the subscript of $U(1)$ in $U(1)_{Y}\otimes SU(2)_{L}\otimes SU(3)_{C}$, and denotes the weak hypercharge that was introduced in Table \ref{tab:fermions}. 
The combined $\mathcal{L}_{U(1)}$ and $\mathcal{L}_{SU(2)}$ looks like
\begin{equation}
\mathcal{L}_{SU(2)\otimes U(1)}=\bar{\Psi}\gamma_{\mu}(i\partial^{\mu}+g\frac{Y}{2}B^{\mu}+g'\frac{\sigma_{i}}{2}W_{i}^{\mu})\Psi-\frac{1}{4}(W_{\mu\nu})_{j}(W^{\mu\nu})_{j}-\frac{1}{4}B_{\mu\nu}B^{\mu\nu} 
\end{equation}                                                
where $g'$ and $g$ denotes the coupling strengths of the three $W_{i}^{\mu}$ fields, and of the field $B^{\mu}$. The last two terms can be interpreted as the kinetic energies of the gauge fields $B^{\mu}$ and $W_{i}^{\mu}$
\subsection*{Quantum chromodynamics} 
\noindent\justify
hej
