\chapter{Event samples and selections}

Now that the LHC, the CMS detector, the collision data and simulated data formats have been presented, the two analyses and the \ptmiss study can be presented. 
This section first presents the datasets used for all parts of the thesis, the triggers used to collect the data and the simulated MC samples. 
Secondly, the physics objects common to the SUSY searches and the \ptmiss performance study are presented. Subsequent to this chapter is the chapter containing the \ptmiss performance study. 
  
\section{Datasets}
\label{sec:samplesObjects}

The data analyzed throughout this thesis consists mainly of the dileptonic streams of 13\TeV pp collision data.  
Additionally, datasets collected with MET and JetHT triggers are analyzed in the context of the FS background prediction method, and single photon samples are used for the \ptmiss study. 
All datasamples are summarized in Table~\ref{tab:datasets}.  
%All events used for the final event counts in the three channels (\EE, \EMu, and \MuMu) were made sure to be assigned to the correct triggering dataset.
%The simulation is normalized to luminosity using cross sections from \url{https://twiki.cern.ch/twiki/bin/viewauth/CMS/SummaryTable1G25ns}.
%MiniAODv2 samples are used consistently for data and MC.

\begin{table}[ht!]
\def\arraystretch{1.2}
    \caption{Datasets used in the slepton/EWK searches and \ptmiss study}
    \label{tab:datasets}
    \begin{center}
        \begin{tabular}{ l}
        \hline\hline 
        \multicolumn{1}{c}{\textbf{Signal events for slepton/EWK search and \ptmiss study}} \\
        \hline
        \texttt{/DoubleEG/Run2016B-03Feb2017\_ver2-v2/MINIAOD}    \\
        \texttt{/DoubleEG/Run2016(C-G)-03Feb2017-v1/MINIAOD}     \\
        \texttt{/DoubleEG/Run2016H-03Feb2017\_ver2-v1/MINIAOD}    \\
        \texttt{/DoubleEG/Run2016H-03Feb2017\_ver3-v1/MINIAOD}    \\
        \texttt{/DoubleMuon/Run2016B-03Feb2017\_ver2-v2/MINIAOD}   \\
        \texttt{/DoubleMuon/Run2016(C-G)-03Feb2017-v1/MINIAOD}  \\
        \texttt{/DoubleMuon/Run2016H-03Feb2017\_ver2-v1/MINIAOD}    \\
        \texttt{/DoubleMuon/Run2016H-03Feb2017\_ver3-v1/MINIAOD}   \\
        \hline        
        \multicolumn{1}{c}{\textbf{Datasets for background prediction}} \\
        \hline
        \texttt{/MuonEG/Run2016B-03Feb2017\_ver2-v2/MINIAOD}    \\
        \texttt{/MuonEG/Run2016(C-G)-03Feb2017-v1/MINIAOD}    \\
        \texttt{/MuonEG/Run2016H-03Feb2017\_ver2-v1/MINIAOD}    \\
        \texttt{/MuonEG/Run2016H-03Feb2017\_ver3-v1/MINIAOD}    \\           
        \texttt{/JetHT/Run2016B-03Feb2017\_ver2-v2/MINIAOD}   \\
        \texttt{/JetHT/Run2016(C-G)-03Feb2017-v1/MINIAOD}   \\
        \texttt{/JetHT/Run2016H-03Feb2017\_ver2-v1/MINIAOD}    \\
        \texttt{/JetHT/Run2016H-03Feb2017\_ver3-v1/MINIAOD}   \\     
        \texttt{/MET/Run2016B-03Feb2017\_ver2-v2/MINIAOD}   \\
        \texttt{/MET/Run2016(C-G)-03Feb2017-v1/MINIAOD}   \\
        \texttt{/MET/Run2016H-03Feb2017\_ver2-v1/MINIAOD}    \\
        \texttt{/MET/Run2016H-03Feb2017\_ver3-v1/MINIAOD}   \\     
        \hline
        \multicolumn{1}{c}{\textbf{Datasets for \ptmiss study}} \\
        \hline
        \texttt{/SinglePhoton/Run2016B-03Feb2017\_ver2-v2/MINIAOD}   \\            
        \texttt{/SinglePhoton/Run2016(C-G)-03Feb2017-v1/MINIAOD}   \\
        \texttt{/SinglePhoton/Run2016H-03Feb2017\_ver2-v1/MINIAOD}    \\
        \texttt{/SinglePhoton/Run2016H-03Feb2017\_ver3-v1/MINIAOD}   \\     
\hline\hline
\end{tabular}
\end{center}
\end{table}

Several Monte Carlo (MC) event generators are used to simulate the background and signal processes in this analysis, and the detector response is simulated using the GEANT package that 
provides a detailed description of the CMS detectora and the event reconstruction is perfromed similarly for data as for simulation. The MC samples are weighted to match the data in the number of 
pileup. The simulation is normalized to luminosity using cross sections from \url{https://twiki.cern.ch/twiki/bin/viewauth/CMS/SummaryTable1G25ns}.
The \texttt{PYTHIA8} \cite{Sjostrand:2006za} package is used for parton showering, hadronization and undelying event simulation with the tune \texttt{CUETP8M1}, as described in Section~\ref{eventReconstructionSimulation}.
In Tables~\ref{tab:MCsamples}, ~\ref{tab:4lMCsamples}, ~\ref{tab:3lMCsamples} and  ~\ref{tab:MCsamplesNonFS}, the various MC simulations used for the different background prediction methods 
are listed.
\begin{table}[ht!]
\def\arraystretch{1.2}
\caption{Simulated SM datasets used for the flavor symmetric (FS) background prediction. All samples are of the
    \texttt{MINIAOD} data format and of the version \texttt{RunIISummer16MiniAODv2-PUMoriond17\_80X\_mcRun2\_asymptotic\_2016\_TrancheIV\_v6/}
    and \texttt{Tune} is short for the pythia8 tune \texttt{CUETP8M1}.}
    \label{tab:MCsamples}
    \begin{center}
        \begin{tabular}{|l l l|}
        \hline \hline    
        Process     &  Dataset  &  $\sigma$ (pb)          \\\hline
        ttbar     &    &            \\
        \scriptsize{\texttt{$t\bar{t}\rightarrow l^{+}\nu b + l^{-}\bar{\nu}\bar{b}$}}     & \scriptsize{\texttt{/TTTo2L2Nu\_TuneCUETP8M2\_ttHtranche3\_13TeV-powheg-pythia8}}     &  \scriptsize{831.76$\times0.1086^{2}\times$9}      \\
        \scriptsize{\texttt{$t\bar{t}\rightarrow l^{-}\bar{\nu}+$jets}} &\scriptsize{\texttt{/TTJets\_SingleLeptFromTbar\_TuneCUETP8M1\_13TeV-madgraphMLM-pythia8}} &   \scriptsize{182.2}      \\
        \scriptsize{\texttt{$t\bar{t}\rightarrow l^{+}\nu+$jets}}     & \scriptsize{\texttt{/TTJets\_SingleLeptFromT\_TuneCUETP8M1\_13TeV-madgraphMLM-pythia8}}     &   \scriptsize{182.2}      \\
        Single Top     &    &            \\
        \scriptsize{\texttt{$W^{+}\rightarrow t\bar{b}$}}     & \scriptsize{\texttt{/ST\_s-channel\_4f\_leptonDecays\_13TeV-amcatnlo-pythia8}}      &  \scriptsize{3.36}       \\
        \scriptsize{\texttt{$\bar{b}\rightarrow \bar{t}W^{+}$}}     & \scriptsize{\texttt{/ST\_tW\_antitop\_5f\_NoFullyHadronicDecays\_13TeV-powheg}}  &   \scriptsize{11.7}      \\
        \scriptsize{\texttt{$b\rightarrow t W^{-}$}}     & \scriptsize{\texttt{/ST\_tW\_top\_5f\_NoFullyHadronicDecays\_13TeV-powheg}}      &  \scriptsize{11.7}       \\
        \scriptsize{\texttt{$q\bar{b}\rightarrow q'\bar{t}$}} & \scriptsize{\texttt{/ST\_t-channel\_antitop\_4f\_inclDecays\_13TeV-powhegV2-madspin-pythia8}}&  \scriptsize{124.0}       \\
        \scriptsize{\texttt{$qb\rightarrow q't$}}     & \scriptsize{\texttt{/ST\_t-channel\_top\_4f\_inclDecays\_13TeV-powhegV2-madspin-pythia8}}      &  \scriptsize{208.0}       \\
        Diboson (FS)    &    &            \\
        \scriptsize{\texttt{$WW\rightarrow l^{+}\nu l^{-}\bar{\nu}$}}     & \scriptsize{\texttt{/WWTo2L2Nu\_13TeV-powheg-pythia8}}  &  \scriptsize{(118.7-3.974)$\times0.1086^{2}\times$9}      \\          
        \scriptsize{\texttt{$gg\rightarrow WW\rightarrow l^{+}\nu l^{-}\bar{\nu}$}}     & \scriptsize{\texttt{/GluGluWWTo2L2Nu\_MCFM\_13TeV}}  &  \scriptsize{(3.974$\times0.1086^{2}\times$9$\times$1.4}      \\          
        \scriptsize{\texttt{$gg\rightarrow H \rightarrow WW$}}     & \scriptsize{\texttt{/GluGluHToWWTo2L2Nu\_M125\_13TeV\_powheg\_JHUgen\_pythia8}}  &  \scriptsize{1.002}      \\          
        \scriptsize{\texttt{$WW$}}     & \scriptsize{\texttt{/WW\_DoubleScattering\_13TeV-pythia8 }}  &  \scriptsize{1.617}      \\          
        \scriptsize{\texttt{$WW$}}     & \scriptsize{\texttt{/WpWpJJ\_EWK-QCD\_TuneCUETP8M1\_13TeV-madgraph-pythia8  }}  &  \scriptsize{0.037}      \\          
        \scriptsize{\texttt{$q\bar{q}\rightarrow l\nu\gamma$}}     & \scriptsize{\texttt{/WGToLNuG\_TuneCUETP8M1\_13TeV-madgraphMLM-pythia8 }}  &  \scriptsize{405.3}      \\          
        Triboson (FS)    &    &            \\
        \scriptsize{\texttt{WWW}}     & \scriptsize{\texttt{/WWW\_Tune\_13TeV-amcatnlo-pythia8}}  & \scriptsize{ 0.209}      \\      
        \scriptsize{\texttt{WW$\gamma$}}     & \scriptsize{\texttt{/WWG\_Tune\_13TeV-amcatnlo-pythia8}}  & \scriptsize{ 0.215}      \\      
        Rare (FS)    &    &            \\
        \scriptsize{\texttt{$t\bar{t}W$}}     & \scriptsize{\texttt{/TTWJetsToLNu\_Tune\_13TeV-amcatnlo-madspin-pythia8}}  &  \scriptsize{0.204}     \\           
        \scriptsize{\texttt{$t\bar{t}W$}}     & \scriptsize{\texttt{/TTWJetsToQQ\_Tune\_13TeV-amcatnloFXFX-madspin-pythia8}}  &  \scriptsize{0.406}     \\           
        \scriptsize{\texttt{$t\bar{t}H$}}     & \scriptsize{\texttt{/ttHToNonbb\_M125\_TuneCUETP8M2\_ttHtranche3\_13TeV-powheg-pythia8}}  &  \scriptsize{0.215}     \\       
        \scriptsize{\texttt{$VH$}}     & \scriptsize{\texttt{/VHToNonbb\_M125\_13TeV\_amcatnloFXFX\_madspin\_pythia8}}  &  \scriptsize{0.952}      \\                           
        \scriptsize{\texttt{$tttt$}}     & \scriptsize{\texttt{/TTTT\_TuneCUETP8M2T4\_13TeV-amcatnlo-pythia8}}  &  \scriptsize{0.009}      \\    
        \scriptsize{\texttt{$W$+jets}}     & \scriptsize{\texttt{/WJetsToLNu\_TuneCUETP8M1\_13TeV-madgraphMLM-pythia8}}  &  \scriptsize{61527}      \\             
\hline\hline
\end{tabular}
\end{center}
\end{table}                                                                                                                                                 

\begin{table}[ht!]
\begin{center}
\def\arraystretch{1.2}
    \caption{Simulated SM datasets used for the ZZ to 4 lepton control regions. All samples are of the
    \texttt{MINIAOD} data format and of the version \texttt{RunIISummer16MiniAODv2-PUMoriond17\_80X\_mcRun2\_asymptotic\_2016\_TrancheIV\_v6/}
and \texttt{Tune} is short for the pythia8 tune \texttt{CUETP8M1}. The k-factor referred to is specified in Subsection~\ref{higherOrderCorr}}
    \label{tab:4lMCsamples}
        \begin{tabular}{|l l l|}
        \hline \hline    
        Process     &  Dataset  &  $\sigma$ (pb)          \\\hline
        ZZ $\rightarrow 4l$     &    &            \\
        \scriptsize{\texttt{$ZZ\rightarrow 4l$}}     & \scriptsize{\texttt{/ZZTo4L\_13TeV\_powheg\_pythia8}}                                 &   \scriptsize{1.256$\times$ k-factor}      \\ 
        \scriptsize{\texttt{$gg\rightarrow H \rightarrow ZZ$}}     & \scriptsize{\texttt{/GluGluHToZZTo4L\_M125\-13TeV\_powheg2\_JHUgenV6\_pythia8}}&   \scriptsize{0.013}      \\ 
        \scriptsize{\texttt{$qq\rightarrow H \rightarrow ZZ$}}     & \scriptsize{\texttt{/VBF\_HToZZTo4L\_M125\_13TeV\_powheg2\_JHUgenV6\_pythia8}}&   \scriptsize{0.001}      \\ 
        \scriptsize{\texttt{$gg \rightarrow ZZ\rightarrow 4e $}}     & \scriptsize{\texttt{/GluGluToContinToZZTo4e\_13TeV\_MCFM701\_pythia8}}  &  \scriptsize{ 0.001586 $\times$ 2.3}      \\    
        \scriptsize{\texttt{$gg \rightarrow ZZ\rightarrow 4\mu$}}     & \scriptsize{\texttt{/GluGluToContinToZZTo4mu\_13TeV\_MCFM701\_pythia8}}  &  \scriptsize{ 0.001586 $\times$ 2.3}      \\    
        \scriptsize{\texttt{$gg \rightarrow ZZ\rightarrow 4\tau$}}     & \scriptsize{\texttt{/GluGluToContinToZZTo4tau\_13TeV\_MCFM701\_pythia8}}  &  \scriptsize{ 0.001586 $\times$ 2.3}      \\    
        \scriptsize{\texttt{$gg \rightarrow ZZ\rightarrow 2e2\tau$}}& \scriptsize{\texttt{/GluGluToContinToZZTo2e2tau\_13TeV\_MCFM701\_pythia8 }}  &  \scriptsize{0.003194 $\times$ 2.3}      \\    
        \scriptsize{\texttt{$gg \rightarrow ZZ\rightarrow 2e2\mu$}} & \scriptsize{\texttt{/GluGluToContinToZZTo2e2mu\_13TeV\_MCFM701\_pythia8 }}  &  \scriptsize{0.003194 $\times$ 2.3}      \\    
        \scriptsize{\texttt{$gg \rightarrow ZZ\rightarrow 2\mu2\tau$}}& \scriptsize{\texttt{/GluGluToContinToZZTo2mu2tau\_13TeV\_MCFM701\_pythia8 }}  &  \scriptsize{0.003194 $\times$ 2.3} \\    
        ZZ $\rightarrow 2l 2\nu$     &    &            \\
        \scriptsize{\texttt{$ZZ\rightarrow 2l2\nu$}}     & \scriptsize{\texttt{/ZZTo2L2Nu\_13TeV\_powheg\_pythia8}}  &  \scriptsize{0.564$\times$ k-factor}      \\
        \scriptsize{\texttt{$gg \rightarrow ZZ\rightarrow 2e2\nu$}} & \scriptsize{\texttt{/GluGluToContinToZZTo2e2nu\_13TeV\_MCFM701\_pythia8 }}  &  \scriptsize{0.001720 $\times$ 2.3}      \\    
        \scriptsize{\texttt{$gg \rightarrow ZZ\rightarrow 2\mu2\nu$}}& \scriptsize{\texttt{/GluGluToContinToZZTo2mu2nu\_13TeV\_MCFM701\_pythia8 }}  &  \scriptsize{0.001720 $\times$ 2.3} \\    
        Others     &    &            \\
        \scriptsize{\texttt{$ZZ\rightarrow 2l2q$}}     & \scriptsize{\texttt{/ZZTo2L2Q\_13TeV\_amcatnloFXFX\_madspin\_pythia8}}  &  \scriptsize{3.28}      \\
        \scriptsize{\texttt{$ZZZ$}}     & \scriptsize{\texttt{/ZZZ\_Tune\_13TeV-amcatnlo-pythia8}}  & \scriptsize{ 0.0139}      \\                               
        \scriptsize{\texttt{$VH$}}     & \scriptsize{\texttt{/VHToNonbb\_M125\_13TeV\_amcatnloFXFX\_madspin\_pythia8}}  &  \scriptsize{0.952}      \\                           
        \scriptsize{\texttt{$q\bar{q}\rightarrow l^{+}l^{-}\gamma$}}     & \scriptsize{\texttt{/ZGTo2LG\_TuneCUETP8M1\_13TeV-amcatnloFXFX-pythia8 }}         &   \scriptsize{123.9}      \\ 
\hline\hline
\end{tabular}
\end{center}
\end{table}                                                                                                                                                                                                                             


\begin{table}[ht!]
\begin{center}
\def\arraystretch{1.2}
    \caption{Simulated SM datasets used for the WZ control regions. All samples are of the
    \texttt{MINIAOD} data format and of the version \texttt{RunIISummer16MiniAODv2-PUMoriond17\_80X\_mcRun2\_asymptotic\_2016\_TrancheIV\_v6/}
    and \texttt{Tune} is short for the pythia8 tune \texttt{CUETP8M1}.}
    \label{tab:3lMCsamples}
        \begin{tabular}{|l l l|}
        \hline \hline    
        Process     &  Dataset  &  $\sigma$ (pb)          \\\hline
        \scriptsize{\texttt{$WZ\rightarrow  l^{+}l^{-} l\nu$}}     & \scriptsize{\texttt{/WZTo3LNu\_TuneCUETP8M1\_13TeV-powheg-pythia8}}  &  \scriptsize{4.429 $\times$ 1.109}      \\             
        \scriptsize{\texttt{$WZ\rightarrow  l^{+}l^{-}2q$}}     & \scriptsize{\texttt{/WZTo2L2Q\_13TeV\_amcatnloFXFX\_madspin\_pythia8}}      &  \scriptsize{5.595 $\times$ 1.109}       \\    
\hline\hline
\end{tabular}
\end{center}
\end{table}                                                                                                                                                                                    

\begin{table}[ht!]
\begin{center}
\def\arraystretch{1.2}
    \caption{Various non flavor symmetric processes. All samples are of the \texttt{MINIAOD} data format and of the version 
        \texttt{RunIISummer16MiniAODv2-PUMoriond17\_80X\_mcRun2\_asymptotic\_2016\_TrancheIV\_v6/} and \texttt{Tune} is short for the pythia8 tune \texttt{CUETP8M1}.}
    \label{tab:MCsamplesNonFS}
        \begin{tabular}{|l l l|}
        \hline \hline    
        Process     &  Dataset  &  $\sigma$ (pb)          \\\hline
        Drell-Yan     &    &            \\
        \scriptsize{\texttt{$Z/\gamma^{*}\rightarrow l^{+}l^{-} (50)$}}     & \scriptsize{\texttt{/DYJetsToLL\_M-50\_Tune\_13TeV-madgraphMLM-pythia8}}      &  \scriptsize{1921.8 $\times$ 3}       \\
        \scriptsize{\texttt{$Z/\gamma^{*}\rightarrow l^{+}l^{-} (10-50)$}}     & \scriptsize{\texttt{/DYJetsToLL\_M-10to50\_Tune\_13TeV-madgraphMLM-pythia8}}  &  \scriptsize{18610}      \\
        Various non FS     &    &            \\
        \scriptsize{\texttt{$t\bar{t}Z$}}     & \scriptsize{\texttt{/TTZToLL\_M-1to10\_TuneCUETP8M1\_13TeV-madgraphMLM-pythia8}}  &  \scriptsize{0.049}     \\                
        \scriptsize{\texttt{$t\bar{t}Z$}}     & \scriptsize{\texttt{/TTZToLLNuNu\_M-10\_Tune\_13TeV-amcatnlo-pythia8}}  &  \scriptsize{0.253}     \\                
        \scriptsize{\texttt{$t\bar{t}Z$}}     & \scriptsize{\texttt{/TTZToQQ\_Tune\_13TeV-amcatnlo-pythia8}}  &  \scriptsize{0.530}     \\              
        \scriptsize{\texttt{$tZq$}}     & \scriptsize{\texttt{/tZq\_ll\_4f\_13TeV-amcatnlo-pythia8}}  &  \scriptsize{0.076}      \\                           
        \scriptsize{\texttt{$tWZ$}}     & \scriptsize{\texttt{/ST\_tWll\_5f\_LO\_13TeV-MadGraph-pythia8}}  & \scriptsize{0.011}      \\     
        \scriptsize{\texttt{WWZ}}     & \scriptsize{\texttt{/WWZ\_Tune\_13TeV-amcatnlo-pythia8}}  & \scriptsize{ 0.165}      \\      
        \scriptsize{\texttt{WZZ}}     & \scriptsize{\texttt{/WZZ\_Tune\_13TeV-amcatnlo-pythia8}}  & \scriptsize{ 0.056}      \\      

\hline\hline
\end{tabular}
\end{center}
\end{table}                                                                                                                                                                              
\subsection{Higher order corrections}
\label{higherOrderCorr}
For the $WZ\rightarrow  l^{+}l^{-} l\nu$ and $WZ\rightarrow  l^{+}l^{-}2q$ processes, a NLO to NNLO correction factor of 1.109 is applied \cite{Grazzini:2016swo}. For the qq$\rightarrow$ZZ 
process, a QCD NLO to NNLO correction factor is applied as a function of generator-level \pt of the diboson system, which is described more in depth in Section~\ref{4l}.
\begin{figure}[!h]
\begin{center}
\begin{tabular}{cc}
\includegraphics[width=0.9\textwidth]{images/slepton/kfactors.pdf} \\
\end{tabular}
\caption{QCD NNLO/NLO k factors for the qq$\rightarrow$ZZ process in generator level variables of the diboson system. }
\label{fig:ZZkfactor}
\end{center}
\end{figure}                                                                                                                                                                                


\section{Triggers}
The trigger selection for the slepton and EWK analyses is driven by the requirement of at least two leptons at the HLT.
More precisely, the signal events in both analyses and the \ptmiss study are collected using dielectron and dimuon triggers.
Additionally, electron-muon triggers are used to collect a sample dominated by \ttbar events for the FS prediction method.

Various requirements on the isolation are imposed on the
The triggers are chosen as a combination of
These triggers are a combination of non-isolated and isolated requirements on the leptons to ensure
The \pt requirements are asymmetric and depend on the flavor composition of the triggers.

Concerning triggers, the selection is dominated by the di-lepton and high \ptmiss signatures. Trigger paths requiring isolated leptons are the main paths. Due to the changes in instantaneous 
luminosity, different di-lepton triggers were active at different times and with varying prescales. This results in the need for a variety of triggers with slighlty different requirements.
Non isolated double lepton paths are included to increase the efficiency in events with large dilepton system \pt. Supporting triggers are used for the study of the 
trigger efficiencies used in the flavor symmetric background prediction method, taken from hadronic events. 
All triggers which are used one way or the other are documented in Table~\ref{tab:triggers}.
\begin{table}[ht!]
\def\arraystretch{1.2}
    \caption{Triggers used in the analysis. The first section are the triggers
    used in most control and signal regions, while the supporting triggers are mostly for the calculation
    of the trigger efficiencies of the signal triggers.}
    \label{tab:triggers}
    \begin{center}
        \begin{tabular}{ l}
        \hline \hline
        \multicolumn{1}{c}{\textbf{Signal triggers} }             \\
        \hline 
        \multicolumn{1}{c}{\texttt{Di-muon triggers} }             \\
        \hline 
        \texttt{HLT\_Mu17\_TrkIsoVVL\_Mu8\_TrkIsoVVL\_v*}         \\
        \texttt{HLT\_Mu17\_TrkIsoVVL\_Mu8\_TrkIsoVVL\_DZ\_v*}      \\
        \texttt{HLT\_Mu17\_TrkIsoVVL\_TkMu8\_TrkIsoVVL\_v*}       \\
        \texttt{HLT\_Mu17\_TrkIsoVVL\_TkMu8\_TrkIsoVVL\_DZ\_v*}     \\
        \texttt{HLT\_Mu27\_TkMu8\_v*}                                \\ 
        \texttt{HLT\_Mu30\_TkMu11\_v*}                               \\
        \hline 
        \multicolumn{1}{c}{\texttt{Di-electron triggers} }             \\
        \hline 
        \texttt{HLT\_Ele17\_Ele12\_CaloIdL\_TrackIdL\_IsoVL\_DZ\_v*}   \\ 
        \texttt{HLT\_Ele23\_Ele12\_CaloIdL\_TrackIdL\_IsoVL\_DZ\_v*}    \\
        \texttt{HLT\_DoubleEle33\_CaloIdL\_GsfTrkIdVL\_v*}               \\
        \texttt{HLT\_DoubleEle33\_CaloIdL\_GsfTrkIdVL\_MW\_v*}               \\
        \hline 
        \multicolumn{1}{c}{\texttt{Electron-muon triggers} }             \\
        \hline 
        \texttt{HLT\_Mu8\_TrkIsoVVL\_Ele17\_CaloIdL\_TrackIdL\_IsoVL\_v*} \\
        \texttt{HLT\_Mu8\_TrkIsoVVL\_Ele23\_CaloIdL\_TrackIdL\_IsoVL\_v*}    \\
        \texttt{HLT\_Mu8\_TrkIsoVVL\_Ele23\_CaloIdL\_TrackIdL\_IsoVL\_DZ\_v*}    \\
        \texttt{HLT\_Mu17\_TrkIsoVVL\_Ele12\_CaloIdL\_TrackIdL\_IsoVL\_v*}    \\
        \texttt{HLT\_Mu23\_TrkIsoVVL\_Ele8\_CaloIdL\_TrackIdL\_IsoVL\_v*}    \\ 
        \texttt{HLT\_Mu23\_TrkIsoVVL\_Ele8\_CaloIdL\_TrackIdL\_IsoVL\_DZ\_v*} \\
        \texttt{HLT\_Mu23\_TrkIsoVVL\_Ele12\_CaloIdL\_TrackIdL\_IsoVL\_v*}    \\ 
        \texttt{HLT\_Mu23\_TrkIsoVVL\_Ele12\_CaloIdL\_TrackIdL\_IsoVL\_DZ\_v*}    \\ 
        \texttt{HLT\_Mu30\_Ele30\_CaloIdL\_GsfTrkIdVL\_v*}            \\
        \texttt{HLT\_Mu33\_Ele33\_CaloIdL\_GsfTrkIdVL\_v*}            \\
        \hline 
        \multicolumn{1}{c}{\textbf{Supporting triggers}} \\
        \hline 
        \texttt{HLT\_PFHT125\_v*}                       \\
        \texttt{HLT\_PFHT200\_v*}                       \\
        \texttt{HLT\_PFHT250\_v*}                       \\
        \texttt{HLT\_PFHT300\_v*}                       \\
        \texttt{HLT\_PFHT350\_v*}                       \\
        \texttt{HLT\_PFHT400\_v*}                       \\
        \texttt{HLT\_PFHT475\_v*}                       \\
        \texttt{HLT\_PFHT600\_v*}                       \\
        \texttt{HLT\_PFHT650\_v*}                       \\
        \texttt{HLT\_PFHT800\_v*}                       \\
        \texttt{HLT\_PFHT900\_v*}                       \\
\hline\hline
\end{tabular}
\end{center}
\end{table}

\section{Physics Objects}
\label{subsec:objects}
Since the principle behind the main background prediction method, described in a separate AN ( ~\cite{CMS_AN_2016-482}), relies on the lepton flavor symmetry of the W decay, the identification and 
isolation requirements of 
the leptons are chosen so that they are as similar as possible between the flavors. This principle is reflected in the selections of the trigger requirement of the leptons of \pt $>$ 23, 17, 12, 
and 8~GeV depending on the exact path. Full efficiency for any of these three values is reached at a \pt of 25(20) GeV for the leading (trailing) lepton , which is
the cut applied to all leptons. One specific selection of this analysis is that not only the electrons are rejected if they appear in the transition region between
the barrel and the endcap, but also the muons. The reason for this is simply that the flavor symmetric background is taken from $e\mu$ events, thus
the necessety of having symmetric cuts not only on the efficiency but also on the fiducial regions. For this reason any lepton within the $|\eta|$ region of 1.4 to 1.6 is rejected.
In addition, selection criteria on jets and \ptmiss, are presented, which all follow standard CMS SUSY recommendations (including the leptons).

\subsection{Electrons}
\label{subsub:electrons}

Table~\ref{tab:electrons} summarizes briefly the most important selection criteria for electrons. Besides kinematical selections, an MVA-trained identification discriminator is used. 
The discriminator is optimized on electrons from prompt W-boson decays in \ttbar versus leptons stemming from so-called ``fakes''\footnote{``Fake'' leptons
are mostly semi-leptonic b-quark decays where $b \rightarrow cW \rightarrow c\ell \nu$} in \ttbar. The used working point for the MVA identification discriminator corresponds to the 
SUSY-PAG-recommended ``tight'' value, developed at the end of 2016. The actual MVA cut value depends on the lepton \pt and $|\eta|$. In each $|\eta|$ bin the lower value is used for electrons
with \pt $> 25$ GeV while the cut decreases linearly from the upper to the lower value for \pt between 15 and 25 GeV. Additionally, conversion rejection cuts are applied. 

\begin{table}[ht!]
\def\arraystretch{1.2}
    \caption{Electron selection criteria.}
    \label{tab:electrons}
    \begin{center}
        \begin{tabular}{ l r}
        \hline \hline
        cut         &  value                             \\ \hline
        \multicolumn{2}{c}{\textbf{Kinematics}}                \\
        \pt         &  $>$ 10 GeV                              \\
        $|\eta|$    &  $<$ 2.4                                 \\
        $|\eta|$    &  $\ni$ [1.4, 1.6]                       \\
        \multicolumn{2}{c}{\textbf{Identification}}                \\
        MVA tight         &  2016 Working point    \\
               \multirow{2}{*}{conversion rejection} & maxLostHits == 0 \\
                & passConversionVeto() \\                          
        \multicolumn{2}{c}{\textbf{Isolation}}                \\
             mini Isolation                 &  $<$ 0.1                         \\
        \multicolumn{2}{c}{\textbf{Impact parameter}}                \\
        $d_{xy}$ & 0.05 \\
        $d_{z }$ & 0.10 \\
        SIP3D    & $< 8$ \\
\hline\hline
\end{tabular}
\end{center}
\end{table}

As isolation variable, the mini-isolation is used which features a shrinking cone-size with increasing
\pt of the lepton. Thus, the cone size in which the PF particles are summed to calculate the relative 
isolation is no longer constant, but a function of the \pt of the lepton
\begin{equation*}
    R = \frac{10.}{min\left[ max \left( \pt, 50 \right), 200 \right]} \quad.
\end{equation*}

For \pt values below 50 GeV, this leads to a constant cone size of 0.2. For \pt values between 50 GeV and 200 GeV, the
cone size shrinks from 0.2 to 0.05 at which it remains for higher \pt leptons.

Corrections to the isolation are applied by subtracting the average energy density 
$\rho$ from the effective geometrical area of the lepton's isolation cone. The variable cone size is taken into account for this correction.

\subsection{Muons}
\label{subsub:muons}

Selection for the muons follow POG recommendations and are summarized in Table~\ref{tab:muons}. All the variables
used are standard variables, and the isolation variable is also mini-isolation.

\begin{table}[ht!]
\def\arraystretch{1.2}
    \caption{Muon selection criteria.}
    \label{tab:muons}
    \begin{center}
        \begin{tabular}{ l r}
        \hline \hline
        cut         &  value                             \\ \hline
        \multicolumn{2}{c}{\textbf{Kinematics}}                \\
        \pt         &  $>$ 10 GeV                              \\
        $|\eta|$    &  $<$ 2.4                                 \\
        $|\eta|$    &  $\ni$ [1.4, 1.6]                       \\
        \multicolumn{2}{c}{\textbf{Identification}}                \\
        \multicolumn{2}{c}{medium Muon ID (POG)}                 \\
        \multicolumn{2}{c}{\textbf{Isolation}}                \\
             mini Isolation                 &  $<$ 0.2                         \\
        \multicolumn{2}{c}{\textbf{Impact parameter}}                \\
        $d_{xy}$ & 0.05 \\
        $d_{z }$ & 0.10 \\
        SIP3D    & $< 8$ \\
\hline\hline
\end{tabular}
\end{center}
\end{table}

\subsection{Lepton pair selection}
\label{subsub:pairsel}
Since there are some events with multiple lepton pairs, it is important to define an unambiguous way of selecting the ``relevant''  opposite-sign, same-flavor lepton pair. The implemented algorithm
selects the two highest \pt leptons which are fully identified and that have a distance between them of 0.1 in $\Delta$R. This is to say, there is no cross-cleaning or prioritization between 
electrons and muons applied, and non-identified leptons (including the crack region) do not enter in the consideration of the lepton pair selection. This is motivated by the expected SUSY signal 
in which real leptons stemming from the decay of heavy particles are expected to be ``clean'' (i.e. well identified) and have rather large \pt. 


\subsection{Isotrack veto}
\label{isotracks}
Events with additional leptons are vetoed by using isolated tracks. These tracks are defined using charged PF candidates with different requirements depending on the flavor. PF electrons 
and PF muons, are required to pass \pt $>$ 5 GeV, $|$dz$|$ $<$ 0.1 cm, as well as being associated to the primary vertex with the requirement of fromPV $>$ 1, and track isolation cuts of 
iso/\pt $<$ 0.2 and iso $<$ 8 GeV. The track isolation sum is computed from all charged PF candidates within a cone of $\Delta$R $<$ 0.3, requiring them to pass $|$dz$|$ $<$ 0.1 cm with respect 
to the primary vertex. Charged PF hadrons are required to pass \pt $>$ 10 GeV, $|$dz$|$ $<$ 0.1 cm, be associated to the PV with fromPV $>$ 1, and track isolation cuts of iso$/$pT $<$ 0.1 and 
iso $<$ 8 GeV. The track  isolation is computed in the same way as for PF leptons above.


\subsection{Jets and $p_{T}^{miss}$}
\label{subsec:jets}
PfCHSjets are considered and standard selections and corrections are followed. The \pt of the jets used for the control regions used for the flavor symmetric background prediction methods 
is 35 GeV and the $|\eta|$ is required to be $<$ 2.4, in various multiplicity depending on the control region. 
The jets are required to be seperated from selected leptons by 0.4 in $\Delta$R. Jet energy corrections are applied, namely the \texttt{Spring16\_23Sep2016V2} set for data and for MC, 
which is the recommended recipe from the JERC group. In the signal region, a veto on jets of \pt greater than 25 GeV is applied, and the same $|\eta|$ requirement as for the jets used 
in the background prediction methods.
The \ptmiss used in this analysis is the Type 1 corrected, which corresponds to the negative of the vectorial sum of all Particle Flow (PF) candidates where the jets are corrected according to the 
jet energy corrections. \ptmiss filters are applied that are designed to reject anomalous \ptmiss events due to misreconstruction, detector noise and non-collision backgrounds. These filters are: 
primary vertex, CSC beam halo, HBHE noise, HBHEiso noise, eebadSC, Ecal TP, bad muon. The detailed
description of these filters can be found in the official page of the MET POG\footnote{https://twiki.cern.ch/twiki/bin/viewauth/CMS/MissingETOptionalFiltersRun2}. 
\subsubsection{\mttwo}
\label{subsec:MT2}
The leptonic \mttwo variable is used to define the signal region. It is a generalization of the transverse mass for pair-produced particles which decay into visible and invisible objects, as 
described in references~\cite{MT2variable,MT2variable2}. The visible objects (in this analysis) are the two selected leptons and the invisible objects either the two LSPs from the slepton decay, 
or neutrinos in the case of di-leptonic \ttbar of leptonic WW production. In events where a pair of particles is produced and where the decay products from each are one visible and one invisible 
particle, the \mttwo has an endpoint at the mass of 
the mother particle. Since the main background in this analysis is \ttbar and leptons from WW production, this variable is efficiently reducing this background with a cut at the W mass, and is 
therefor used to define the signal regions in this analysis.
\subsection{Event selection}
\label{subsec:selectionSummary}
To summarize, the following list gives all the details of the minimal event selection used for the analysis.
\begin{itemize}
    \item two selected leptons of opposite charge with \pt $>$ 25 GeV for the leading and \pt $>$ 20 GeV for the trailing lepton
    \item \dr between the two leptons of $>$ 0.1
    \item the two highest \pt leptons are selected for this pair
    \item \mll $>$ 20 GeV and a wide Z veto where events with 76 $<$  \mll $<$ 106 GeV are rejected. 
    \item \ptmiss $>$ 100 GeV
\end{itemize}                                                                                                                                  
