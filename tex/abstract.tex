% Write in only the text of your abstract, all the extra heading jargon is automatically taken care of
\begin{abstract}
This document presents two searches for physics beyond the Standard Model (SM), each using $35.9\mathrm{fb^{-1}}$ of proton--proton collision data collected with the CMS detector at a center-of-mass energy of 13 TeV, at the CERN Large Hadron Collider (LHC). 
The two searches for new phenomena is targeting electroweak production of Supersymmetric (SUSY) particles, so called Charginos, Neutralinos and sleptons, in a production mode that results in two leptons of opposite-sign and same-flavor, large missing transverse momentum, $\mathrm{p_{T}^{miss}}$.  
\\
The document contains a summary of the theoretical framework that make up the SM and SUSY, along with a comprehensive description of the CMS experiment at the LHC accelerator complex. 
The two searches presented in this thesis both target the production of electroweak SUSY particles, but are divided into two types, according to the production mode. 
The search for Charginos and Neutralinos result in final states where two or more jets resulting from hadronization are produced, while the search for the direct production of sleptons is chrarcterized by the fact that no hadronization is expected, and thus results in a final state without any jets. The search strategies thus differ slightly, and the two strategies are presented, along with a description of the SM background processes that govern these final states. 

Since no excess of collision data is observed with respect to the predicted SM backgrounds in neither of the searches, a statistical interpretation of the results yielding upper limits in the production cross sections on the SUSY particles, is performed. These limits greatly extend the limits set using 8 TeV collision data during the LHC Run 1. 

Concluding remarks commenting on the current absence of evidence for physics beyond the SM (BSM) are given, and an outlook highlighting the unprecedented instantaneous luminosity expected at the LHC, and the window of opportunity for searches for BSM physics that it presents. 
\end{abstract}
