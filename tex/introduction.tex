\chapter{INTRODUCTION} \label{intro}
\noindent\justify
The Standard Model (SM) of particle physics is an exceptionally successful theory shown to accurately predict the behavior of the particles and forces that make up the most basic constituents of the Universe. 
It manages to correctly describe all of the known forces except for the graviational. 
The SM predicts that massive particles of the theory acquire their mass by interacting with a scalar particle called the Higgs boson. 
On July 4, 2012 two collaborations at the Large Hadron Collider (LHC), the A Toroidal LHC Apparatus (ATLAS) and Compact Muon Solenoid (CMS), announced the discovery of a new boson of a mass of 125\GeV with properties similar to the SM Higgs. 
This boson had been theorized since the 1960's and its discovery has futher grounded the SM as the number one theory that explains all particles and interactions. 
\newpara
\noindent\justify
But in order for the SM to be considered a complete theory, it needs to be able to explain some outstanding features. 
Among these features, is why the lightest particle, the neutrino, and the heaviest particle, the top quark, span multiple orders of magnitude in mass. 
Further, a vast portion of the Universe is predicted to consist of an as-of-yet undetected, non-luminous form of matter, known as dark matter. 
The formulation of the SM is unable to provide an explanation of the origin of this dark matter.  
In order to account for these and other phenomena, possible theoretical extensions of the SM are proposed, of which Supersymmetry (SUSY) is by far the theory that has attracted most popularity. 
The popularity originates from the theorys ability to solve many of the problems of the SM, by doubling the particle content of the SM. 
The only problem with SUSY, is that it has not been discovered yet. 
\newpara
\noindent\justify
This work presents a number of searches for SUSY particles, that all have in common that they produce two electrons or two muons of opposite charge. 
The analyses make use of the data from proton-proton collisions recorded by the CMS detector at the LHC in Geneva, Switzerland during 2016. 
The collision energy amounts to the unprecedented value of 13\TeV in center-of-mass, and the amuont of data, the integrated luminosty, corresponds to $35.9\fbinv$.   
\newpara
\noindent\justify
This thesis is structured in four parts. 
The first part consists of three chapters that introduces the theroetical framework that lays the foundation and motivation of this thesis. 
The SM of particle physics is presented, its successes and shortcomings are discussed, that leads to the motivation of an extension of the SM, SUSY, which is introduced in the subsequent chapter. 
The first part concludes with a summary of how searches for SUSY can be performed using hadron colliders. 
\newpara
\noindent\justify
The second part of the thesis contains a description of the experimental setup needed to conduct searches for SUSY. 
The LHC accelerator complex is introduced, followed by an overview of the layout of the CMS experiment. 
The means by which the data is collected is presented, along with a description of the algorithms that reconstruct the physics objects.  
\newpara
\noindent\justify
The third part of the thesis outlines the strategy by which searches of measurements are performed. 
Searches for SUSY, that often contain the production of particles that does not interact with the detector material, rely on a variable known as missing transverse momentum. 
This variable is be used to measure particles that escape detection, and is reconstructed as the negative vectorial sum of all the visible objects in the event. 
The first chapter of this part summarizes the excellent performance of the missing transverse momentum reconstruction algorithms used in CMS, which is an essential ingredient in any SUSY search. 
The rest of the part contains a description of how the searches for SUSY particles is performed, including the object selections, signal region construction and how the known SM processes that are backgrounds for the searches are estimated. 
Finally, the statistical analysis used to interpret the results is presented.   
\newpara
\noindent\justify
The final part of the thesis contains the actual searches for SUSY that I contributed to as a PhD student at ETH Z\"{u}rich. 
Three searches are presented, namely a search for colored superpartners, electroweak superpartners, and a search for direct production of the superpartners of the SM leptons. 
To date, the results provide the most stringent limits on the electroweak superpartners, and provide exclusion on a number of SUSY particles.   



