\chapter{INTRODUCTION} \label{intro}

The Standard Model (SM) of particle physics is an extremely successful theory shown to correctly predict the behavior of the particles and forces which make up the most basic constituents of all known matter. After the prediction of the existence of a spin-0 boson with ability to explain how all particles acquire mass in the 60's, and the subsequent discovery of the Higgs boson in 2012 by the ATLAS and CMS experiments, the SM is established as the most precise theory to explain matter. 
But in order for the SM to be a complete theory, it needs to be able to explain some interesting features. Among these features, is why the lightest particle, the neutrino, and the heaviest particle, the top quark, span 3 orders of magnitude in mass. And further, the SM is not able to explain the origin of dark matter. In order to account for these phenomena, possible theoretical extensions of the SM are proposed, of which Supersymmetry is by far the theory that has attracted most popularity. The popularity     
In particular, the SM predicts that the massive particles of the theory acquire their mass by interacting with a scalar particle called the Higgs boson \cite{higgs1,higgs2,higgs3,qftam}. On July 4, 2012 two collaborations at the Large Hadron Collider (LHC), the A Toroidal LHC Apparatus (ATLAS) and Compact Muon Solenoid (CMS), announced the discovery of a new boson at 125 GeV with properties similar to the Standard Model Higgs \cite{atlasdiscovery,cmsdiscovery2012,cmsdiscovery2013}. This discovery was fueled by the investigation into the Higgs decays to the vector bosons ZZ and ${\rm \gamma\gamma}$. 

